
\input{../preamble-tmp.tex}

\title%
{Clases en C++ - Const}

\subject{Clases en C++ - Const}

\begin{document}

\begin{frame}[noframenumbering,plain]
   \titlepage
\end{frame}


~% macro content() %~
\section{Constantes}
\subsection{Constantes}
\begin{frame}[fragile]{M\'etodos constantes: no modifican al objeto}
                   ~% if interactive is off %~
        \begin{lstlisting}[style=normal,firstnumber=1,linebackgroundcolor={%
                 \btLstHLB<1>{9-11}%
         }]~% else %~
        \begin{lstlisting}[style=normal,firstnumber=1]~% endif %~
struct Vector {
  int *data;
  int size;

  void set(int pos, int val) {
      this->data[pos] = val;
  }

  int get(int pos) const {
      return this->data[pos];
  }

  /* ... */
};
        \end{lstlisting}
\end{frame}
\note[itemize] {
\item Un m\'etodo constante es un m\'etodo que no modifica el estado interno del objeto. Esto es, no cambia ning\'un atributo ni llama a ning\'un m\'etodo salvo que este sea tambi\'en constante.
\item Sirve para detectar errores en el c\'odigo en tiempo de compilaci\'on: si un m\'etodo no modifica el estado deber\'ia poderse ponerle la keyword \lstinline[style=normal]!const!; si el compilador falla es por que hay un bug en el c\'odigo y nuestra hip\'otesis de que el m\'etodo no cambiaba el estado interno del objeto es err\'onea.
}

\begin{frame}[fragile]{Objetos constantes}
        \begin{lstlisting}[style=normal,firstnumber=17]
void f() {
  Vector v(5);

  v.set(0, 1); // no const
  v.get(0); // const
}

        \end{lstlisting}
        \pause
                   ~% if interactive is off %~
        \begin{lstlisting}[style=normal,firstnumber=24,linebackgroundcolor={%
                 \btLstHLR<2>{27}%
         }]~% else %~
        \begin{lstlisting}[style=normal,firstnumber=24]~% endif %~
void f() {
  const Vector v(5); // objeto constante

  v.set(0, 1); // no const
  v.get(0); // const
}
        \end{lstlisting}
\end{frame}

\begin{frame}[fragile]{Const como promesa}
                   ~% if interactive is off %~
        \begin{lstlisting}[style=normal,firstnumber=17,linebackgroundcolor={%
                 \btLstHLB<1>{23}%
                 \btLstHLR<1>{24}%
         }]~% else %~
        \begin{lstlisting}[style=normal,firstnumber=17]~% endif %~
void f() {
  Vector v(5);

  g(v);
}

void g(const Vector &v) {
  v.set(0, 1); // no const
  v.get(0); // const
}
        \end{lstlisting}
\end{frame}
\note[itemize] {
\item Es comun recibir par\'ametros constantes. La funci\'on promete que no va a cambiar al objeto recibido como par\'ametro.
}

\begin{frame}[fragile]{Atributos constantes}
                   ~% if interactive is off %~
        \begin{lstlisting}[style=normal,firstnumber=1,linebackgroundcolor={%
                 \btLstHLB<1>{2,3}%
                 \btLstHLB<2>{2,3,10}%
                 \btLstHLB<3>{2,3,6}%
         }]~% else %~
        \begin{lstlisting}[style=normal,firstnumber=1]~% endif %~
struct Vector {
  int * const data; // no confundir con int const * data;
  const int size; // equivalente a int const size;

  void set(int pos, int val) {
      this->data[pos] = val;
  }

  int get(int pos) const {
      return this->data[pos];
  }

  /* ... */
};
        \end{lstlisting}
\end{frame}
\note[itemize] {
\item Tambi\'en podemos tener atributos constantes. Estos toman un valor cuando se crean y lo mantienen durante toda la vida del objeto.
\item Peque\~na aclaraci\'on: \lstinline[style=normal]!const int! y \lstinline[style=normal]!int const! son equivalentes asi como tambi\'en \lstinline[style=normal]!const int *! y \lstinline[style=normal]!int const *!. Sin embargo es distinto \lstinline[style=normal]!int * const!. Confuso?
\item \lstinline[style=normal]!const int * p! se lee como "\lstinline[style=normal]!p! es un puntero; a \lstinline[style=normal]!int!; constante" mientras que \lstinline[style=normal]!int * const p! se lee como "\lstinline[style=normal]!p! es constante; puntero; a \lstinline[style=normal]!int!". El primero apunta a \lstinline[style=normal]!int!s constantes mientras que el segundo es el puntero quien es constante.
}

\subsection{Initialization}
                   ~% if interactive is off %~
\begin{frame}[fragile]{Member Initialization List}
        \begin{lstlisting}[style=normal,firstnumber=1,linebackgroundcolor={%
                 \btLstHLB<1>{5}%
                 \btLstHLB<2>{6-8}%
         }]
struct Vector {
  int *data;
  int size;

  Vector(int size) {
    // atributos ya construidos;  aca solo los re-asigno
    this->data = malloc(size*sizeof(int));
    this->size = size;
  }
        \end{lstlisting}
        \pause
        \pause
        \begin{lstlisting}[style=normal,firstnumber=1,linebackgroundcolor={%
                 \btLstHLB<3>{5-6}%
         }]
struct Vector {
  int *data;
  int size;

  Vector(int size) : data(malloc(size*sizeof(int))),
                     size(size) {

  }
        \end{lstlisting}
\end{frame}
\note[itemize] {
\item Al ejecutarse el cuerpo del constructor todos sus atributos ya estan creados.
\item Si se necesita construir alguno o todos sus atributos con par\'amentros especiales hay que usar la member initialization list.
\item Esto es \'util no s\'olo para crear objetos que no pueden cambiar una vez construidos (como los atributos \lstinline[style=normal]!const! y las referecias) sino que tambi\'en es necesario si queremos construir otros objetos con par\'ametros custom, sean nuestros atributos o nuestros ancestros (herencia).
}
                   ~% endif %~

\begin{frame}[fragile]{Inicializaci\'on de atributos constantes}
                   ~% if interactive is off %~
        \begin{lstlisting}[style=normal,firstnumber=1,linebackgroundcolor={%
                 \btLstHLR<1>{2,7}%
                 \btLstHLR<2>{3,8}%
         }]~% else %~
        \begin{lstlisting}[style=normal,firstnumber=1]~% endif %~
struct Vector {
  int * const data;
  const int size;

  Vector(int size) {
    // atributos ya construidos;  aca solo los re-asigno
    this->data = malloc(size*sizeof(int));
    this->size = size;
  }
        \end{lstlisting}
        \pause
        \pause
                   ~% if interactive is off %~
        \begin{lstlisting}[style=normal,firstnumber=1,linebackgroundcolor={%
                 \btLstHLB<3>{2,5}%
                 \btLstHLB<4>{3,6}%
         }]~% else %~
        \begin{lstlisting}[style=normal,firstnumber=1]~% endif %~
struct Vector {
  int * const data;
  const int size;

  Vector(int size) : data(malloc(size*sizeof(int))),
                     size(size) {

  }
        \end{lstlisting}
\end{frame}
\note[itemize] {
\item La member initialization list es el \'unico lugar para inicializar atributos constantes y referencias.
}

\begin{frame}[fragile]{Inicializaci\'on de atributos no-default}
                   ~% if interactive is off %~
    \begin{lstlisting}[style=normal,firstnumber=1,linebackgroundcolor={%
                 \btLstHLR<1>{5}%
         }]~% else %~
    \begin{lstlisting}[style=normal,firstnumber=1]~% endif %~
struct DoubleVector {
  Vector fg;
  Vector bg;

  DoubleVector(int size) {
      // fg, bg??
  }
}
        \end{lstlisting}
        \pause
                   ~% if interactive is off %~
    \begin{lstlisting}[style=normal,firstnumber=1,linebackgroundcolor={%
                 \btLstHLB<2>{5}%
         }]~% else %~
    \begin{lstlisting}[style=normal,firstnumber=1]~% endif %~
struct DoubleVector {
  Vector fg;
  Vector bg;

  DoubleVector(int size) : fg(size), bg(size) {
  }
}
        \end{lstlisting}
\end{frame}
\note[itemize] {
\item La member initialization list es el \'unico lugar para inicializar atributos que son objetos que no tienen un constructor por default o sin par\'ametros.
}

\begin{frame}[fragile]{Delegating constructors}
    \begin{lstlisting}[style=normal,firstnumber=1]
struct DoubleVector {
  DoubleVector(int size) : fg(size), bg(size) { }

  DoubleVector(int size, int val) : fg(size), bg(size) {
      for (int i = 0; i < size; ++i) {
          fg.set(i, val);
          bg.set(i, val);
      }
        \end{lstlisting}
        \pause
    \begin{lstlisting}[style=normal,firstnumber=1]
struct DoubleVector {
  DoubleVector(int size) : fg(size), bg(size) { }

  DoubleVector(int size, int val) : DoubleVector(size) {
      for (int i = 0; i < size; ++i) {
          fg.set(i, val);
          bg.set(i, val);
      }
        \end{lstlisting}
\end{frame}
\note[itemize] {
\item La member initialization list permite llamar a otro constructor para delegarle parte de la construcci\'on del objeto. Esto permite reutilizar c\'odigo entre los constructores.
}


~% endmacro %~

~{ content() }~

\end{document}


