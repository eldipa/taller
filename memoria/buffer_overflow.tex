\input{../preamble-tmp.tex}

\title%
{Memoria en C/C++ - Errores de Memoria}


\subject{Memoria en C/C++ - Errores de Memoria}


\begin{document}

\begin{frame}[noframenumbering,plain]
   \titlepage
\end{frame}


~% macro content() %~
\section{Buffer overflows}
\begin{frame}[fragile]{Smash the stack for fun and profit}
         \begin{lstlisting}[style=normal]
// compilar con el flag -fno-stack-protector
#include <stdio.h>

int main(int argc, char *argv[]) {
    int cookie = 0;
    char buf[10];

    printf("buf: %08x cookie: %08x\n", buf, &cookie);
    gets(buf);

    if (cookie == 0x41424344) {
        printf("You win!\n");
    }

    return 0;
} // Insecure Programming
         \end{lstlisting}
\end{frame}
\note[itemize] {
\item Es claro que al inicializar \lstinline[style=normal]!cookie! a cero nunca se va a imprimir \lstinline[style=normal]?"You win!"?... o si?
\item \lstinline[style=normal]!gets! lee de la entrada est\'andar hasta encontrar un \lstinline[style=normal]!'\\n'! y lo que leea lo escribira en el buffer \lstinline[style=normal]!buf!. Pero si el input es m\'as grande que el buffer, \lstinline[style=normal]!gets! escribira por fuera de este y sobreescribira todo el stack lo que se conoce como Buffer Overflow.
\item Para hacer que el programa entre al \lstinline[style=normal]!if! e imprima \lstinline[style=normal]?"You win!"? se debe forzar a un buffer overflow con un input crafteado:
\item Debe tener 10 bytes de m\'inima para ocupar el buffer \lstinline[style=normal]!buf!.
\item Posiblemente deba tener algunos bytes adicionales para ocupar el posible espacio de padding usado para alinear las variables.
\item Luego se debe escribir los 4 bytes que sobreescribiran \lstinline[style=normal]!cookie! pero cuidado, dependiendo de la arquitectura y flags del compilador \lstinline[style=normal]!sizeof(int)! puede no ser 4.
\item Suponiendo que sean 4 bytes, hay que escribir el n\'umero \lstinline[style=normal]!0x41424344! byte a byte y el orden dependera del endianess: \lstinline[style=normal]!"ABCD"! en big endian, \lstinline[style=normal]!"DCBA"! en little endian.
}

\begin{frame}[fragile]{Buffer overflow}{}
   \begin{itemize}
      \item Funciones inseguras que no ponen un l\'imite en el tama\~no del buffer que usan. No usarlas!
         \begin{lstlisting}[style=normal]
gets(buf);
strcpy(dst, src);
         \end{lstlisting}

     \item Reemplazarlas por funciones que s\'i permiten definir un l\'imite, pero es responsabilidad del programador poner un valor coherente! 
         \begin{lstlisting}[style=normal]
getline(buf, max_buf_size, stream);
strncpy(dst, src, max_dst_size);
         \end{lstlisting}
   \end{itemize}
\end{frame}
\begin{frame}[fragile]{Challenge: hacer que el programa imprima "You win!"}
         \begin{lstlisting}[style=normal]
// compilar con el flag -fno-stack-protector
#include <stdio.h>

int main(int argc, char *argv[]) {
    int cookie = 0;
    char buf[10];

    printf("buf: %08x cookie: %08x\n", buf, &cookie);
    gets(buf);

    if (cookie == 0x41424344) {
        printf("You loose!\n");
    }

    return 0;
} // Insecure Programming
         \end{lstlisting}
\end{frame}

~% endmacro %~

~{ content() }~

\end{document}


