\input{../preamble-tmp.tex}

\title%
{Memoria en C/C++}

\subject{Memoria en C/C++}

\begin{document}

\begin{frame}
   \titlepage
\end{frame}

\begin{frame}{De qu\'e va esto?}
   \tableofcontents
   % You might wish to add the option [pausesections]
\end{frame}

~% from "align_size.tex" import content as align_size_content with context %~
~{ align_size_content() }~

~% from "segments.tex" import content as segments_content with context %~
~{ segments_content() }~

~% from "pointers.tex" import content as pointers_content with context %~
~{ pointers_content() }~

~% from "buffer_overflow.tex" import content as buffer_overflow_content with context %~
~{ buffer_overflow_content() }~

\appendix
\section<presentation>*{\appendixname}
\subsection<presentation>*{Referencias}

\begin{frame}[allowframebreaks]
   \frametitle<presentation>{Referencias}

   \begin{thebibliography}{10}

         \beamertemplatebookbibitems
         % Start with overview books.

      \bibitem{Stroustrup}
         Bjarne Stroustrup.
         \newblock {\em The C++ Programming Language}.
         \newblock Addison Wesley, Fourth Edition.

         \beamertemplatearticlebibitems

      \bibitem{man page: gets strcpy htons qsort}
         man page: gets strcpy htons qsort

      \bibitem{Insecure Programming}
         Insecure Programming

         % Followed by interesting articles. Keep the list short.

   \end{thebibliography}
\end{frame}

\end{document}


