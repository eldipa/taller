\input{../preamble-tmp.tex}

\title%
{Memoria en C/C++ - Punteros}


\subject{Memoria en C/C++ - Punteros}


\begin{document}

\begin{frame}
   \titlepage
\end{frame}

~% if interactive is off %~
\begin{frame}{De qu\'e va esto?}
   \tableofcontents
   % You might wish to add the option [pausesections]
\end{frame}
~% endif %~

~% macro content() %~
\section{Punteros}
% simples
\begin{frame}[fragile]{Punteros}
         \begin{lstlisting}[style=normal]
int *p;    // p es un puntero a int
           // (p guarda la direccion de un int)

int i = 1;
p = &i;    // &i es la direccion de la variable i

*p = 2;    // *p dereferencia o accede a la memoria
           // cuya direccion esta guardada en p

/* i == 2 */

         \end{lstlisting}
         \begin{lstlisting}[style=normal]

char buf[512];
write(&buf[0], 512);

         \end{lstlisting}
\end{frame}

% aritmetica de punteros
\begin{frame}[fragile]{Aritm\'etica de punteros}
         \begin{lstlisting}[style=normal]
int a[10];
int *p;

p = &a[0];

*p       // a[0]
*(p+1)   // a[1]


int *p;
p+1      // movete sizeof(int) bytes (4)

char *c;
c+2      // movete 2*sizeof(char) bytes (2)

         \end{lstlisting}
\end{frame}
\note[itemize] {
\item La notaci\'on de array (indexado) y la aritm\'etica de punteros son escencialmente lo mismo.
\item La aritm\'etica de punteros se basa en el tama\~no de los objetos a los que se apunta al igual que el indexado de un array.
}

% punteros a funcion
\begin{frame}[fragile]{Punteros a funciones (al code segment)}
         \begin{lstlisting}[style=normal]
int g(char) {}

int (*p)(char);
p = &g;
         \end{lstlisting}
\pause
         \begin{lstlisting}[style=normal]
#include <stdlib.h>
void qsort(void *base,
           size_t nmemb,
           size_t size,

           int (*cmp)(const void *, const void *)
          );

int cmp_personas(const void* a, const void* b) {
    struct Persona *pa = (struct Persona*)a;
    struct Persona *pb = (struct Persona*)b;

    return pa->edad - pb->edad;
}

         \end{lstlisting}
\end{frame}


~% if interactive is off %~
\begin{frame}[fragile]{Como leer la bizarra notaci\'on de punteros en C/C++}
         \begin{lstlisting}[style=dimmided]
@/* Ejemplo 1 */
char *a[10];@@
      a        // "a"
     *a        // "a" apunta a
char *a        // "a" apunta a char
char *a[10];   // "a" apunta a char (10 de esos)

char *a[10];   // "a" es un array de 10 de esos, o sea
               // "a" es un array de 10 punteros a char
@
         \end{lstlisting}
\end{frame}
~% else %~
\begin{frame}[fragile,t]{Como leer la bizarra notaci\'on de punteros en C/C++}
         \begin{lstlisting}[style=normal]
/* Ejemplo 1 */
char *a[10];
         \end{lstlisting}
\end{frame}
~% endif %~


~% if interactive is off %~
\begin{frame}[fragile]{Como leer la bizarra notaci\'on de punteros en C/C++}
         \begin{lstlisting}[style=dimmided]
@/* Ejemplo 2 */
char (*c)[10];@@
       c       // "c"
      *c       // "c" apunta a
     (*c) == X // llamemos "X" a (*c) temporalmente
@@
char X[10];
char X[10];    // "X" es un char (10 de esos)

char X[10];    // "X" es un array de 10 char
@@char (*c)[10]; // "c" apunta a un array de 10 char
@
         \end{lstlisting}
\end{frame}
~% else %~
\begin{frame}[fragile,t]{Como leer la bizarra notaci\'on de punteros en C/C++}
         \begin{lstlisting}[style=normal]
/* Ejemplo 2 */
char (*c)[10];
         \end{lstlisting}
\end{frame}
~% endif %~


~% if interactive is off %~
\begin{frame}[fragile]{Como leer la bizarra notaci\'on de punteros en C/C++}
         \begin{lstlisting}[style=dimmided]
@/* Ejemplo 3: modo dios */
char (*f)(int)[10];@@
       f            // "f"
      *f            // "f" apunta a
     (*f) == X
@@
char X(int)[10];@@
char X(int)         // es la firma de una funcion,
                    // asi que vuelvo un paso para atras
char (*f)(int)      // entonces esto es un puntero a funcion
                    // cuya firma recibe un int y retorna
                    // un char
@@
char (*f)(int)[10]; // puntero a funcion, 10 de esos
char (*f)(int)[10]; // f es un array de 10 punteros a funcion,
                    // que reciben un int y retornan un chars
@
         \end{lstlisting}
\end{frame}
~% else %~
\begin{frame}[fragile,t]{Como leer la bizarra notaci\'on de punteros en C/C++}
         \begin{lstlisting}[style=normal]
/* Ejemplo 3 */
char (*f)(int)[10];
         \end{lstlisting}
\end{frame}
~% endif %~


% typedef
\begin{frame}[fragile]{Simplificando la notaci\'on}{}
         \begin{lstlisting}[style=normal]
        char *X[10];   // la variable "X" es un array de
                       // 10 punteros a char

         \end{lstlisting}
         \begin{lstlisting}[style=normal]
typedef char *X[10];   // el tipo "X" es un array de
                       // 10 punteros a char

X my_array;            // es una alias, decir "X" es como decir
char *my_array[10];    // "array de 10 punteros a char"

         \end{lstlisting}
\end{frame}

\begin{frame}[fragile]{Simplificando la notaci\'on}{}
Si quiero una variable que sea un array de punteros a funci\'on que no reciban ni retornen nada?
         \begin{lstlisting}[style=normal]
        void (*X)();   // la variable "X" es un puntero a
                       // funcion

         \end{lstlisting}
         \begin{lstlisting}[style=normal]
typedef void (*X)();   // el tipo "X" es un puntero a
                       // funcion

X f[10];               // f es una array de 10 X, entonces
                       // f es una array de 10 punteros
                       // a funcion
         \end{lstlisting}
\end{frame}

~% endmacro %~

~{ content() }~


\end{document}


