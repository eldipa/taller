
\input{../preamble-tmp.tex}

\title%
{Pasaje de objetos en C++ - Pasajes}

\subject{Pasaje de objetos en C++ - Pasajes}

\tikzset{
  pointer/.style={->, dashed, draw, black, line width=0.03cm},
}

\tikzset{
 nodeinvisible/.style={opacity=.01},
 nodevisible on/.style={alt={#1{}{nodeinvisible}}},
 arrowinvisible/.style={opacity=.01},
 arrowvisible on/.style={alt={#1{}{arrowinvisible}}},
 alt/.code args={<#1>#2#3}{%
   \alt<#1>{\pgfkeysalso{#2}}{\pgfkeysalso{#3}} % \pgfkeysalso doesn't change the path
 },
}

\begin{document}

\begin{frame}[noframenumbering,plain]
   \titlepage
\end{frame}


~% macro content() %~
\section{Pasaje de objetos}
\subsection{Pasaje por referencia}
\begin{frame}[fragile]{C\'odigo base}
        \begin{lstlisting}[style=normal,firstnumber=1]
struct Vector {
  int *data;
  int size;

  Vector(int size) {  // create
    this->data = (int*)malloc(size*sizeof(int));
    memset(this->data, 0, size*sizeof(int));
    this->size = size;
  }

  ~Vector() { // destroy
    free(this->data);
  }
};
        \end{lstlisting}
\end{frame}

\begin{frame}[fragile]{Pasaje por referencia usando punteros}
    \begin{columns}[T]
      \begin{column}{.45\linewidth}
                   ~% if interactive is off %~
        \begin{lstlisting}[style=normal,firstnumber=1,linebackgroundcolor={%
                 \btLstHLB<1>{3}%
                 \btLstHLB<2>{4,9}%
                 \btLstHLB<3>{11}%
                 \btLstHLB<4>{6}%
                 \btLstHLB<5>{7}%
         }]~% else %~
        \begin{lstlisting}[style=normal,firstnumber=1]~% endif %~
// con punteros
int foo() {
    Vector v(3);
    bar(&v);

    v.get(0);
}

void bar(Vector* w) {
  for (int i = 0; /*...*/)
    w->set(i, 1);
}
         \end{lstlisting}

      \end{column}
      \begin{column}{.5\linewidth}
                   ~% if interactive is off %~
         \begin{tikzpicture}[cell/.style={rectangle,draw=black},
            space/.style={minimum height=1.5em,matrix of nodes,row sep=-\pgflinewidth,column sep=-\pgflinewidth,column 1/.style={font=\ttfamily}},text depth=0.5ex,text height=2ex,nodes in empty cells]
            \matrix (stackname)[space, column 1/.style={font=\ttfamily},column 2/.style={font=\ttfamily,nodes={minimum width=5.5em}}] 
            {
                   & {stack}  \\
            };
            \matrix (foo)[below =0.1em of stackname, space, column 1/.style={font=\ttfamily},column 2/.style={nodes={cell,minimum width=5.5em}},column 3/.style={nodes={cell,minimum width=2em}},column 4/.style={nodes={cell,minimum width=2em}},column 5/.style={nodes={cell,minimum width=2em}}] 
            {
                {\visible<1->{v}}   & {\only<1-4>{v.data}\only<5>{XX}}  \\
                                    & {\only<1-4>{v.size = 3}\only<5>{XX}}  \\
            };
             \matrix (bar)[nodevisible on={<2-3>}, below =0.1em of foo, space, column 1/.style={font=\ttfamily},column 2/.style={nodes={cell,minimum width=5.5em}},column 3/.style={nodes={cell,minimum width=2em}},column 4/.style={nodes={cell,minimum width=2em}},column 5/.style={nodes={cell,minimum width=2em}}] 
            {
                {\visible<2->{w}}   & {\visible<2->{\&v}}  \\
            };

            \matrix (heapname)[right=2em of foo, space, column 1/.style={font=\ttfamily},column 2/.style={font=\ttfamily,nodes={minimum width=5.5em}}] 
            {
                   & {heap}  \\
            };
             \matrix (heap)[below =0.1em of heapname, space, column 1/.style={nodes={cell,minimum width=2em}},column 2/.style={nodes={cell,minimum width=2em}},column 3/.style={nodes={cell,minimum width=2em}},column 4/.style={nodes={cell,minimum width=2em}},column 5/.style={nodes={cell,minimum width=2em}}] 
            {
                       {\only<1-2>{0}\only<3-4>{1}\only<5>{XX}} & {\only<1-2>{0}\only<3-4>{1}\only<5>{XX}} & {\only<1-2>{0}\only<3-4>{1}\only<5>{XX}} \\
            };

             \draw [->, pointer, arrowvisible on={<2-3>}] (bar-1-2.west) to[out=115,in=245] (foo-1-2.west);
             \draw [->, pointer] (foo-1-2.east) to (heap.west);


         \end{tikzpicture}
        \begin{onlyenv}<1>
        \begin{lstlisting}[style=normal,firstnumber=5]
  Vector(int size) {  // create
    data = malloc(..);
    memset(data, 0 ..);
    this->size = size;
  }
        \end{lstlisting}
        \end{onlyenv}
        \begin{visibleenv}<5>
        \begin{lstlisting}[style=normal,firstnumber=10]
  ~Vector() { // destroy
    free(data);
  }
        \end{lstlisting}
        \end{visibleenv}
                   ~% endif %~
      \end{column}
   \end{columns}
\end{frame}

\begin{frame}[fragile]{Pasaje por referencia usando referencias}
    \begin{columns}[T]
      \begin{column}{.45\linewidth}
                   ~% if interactive is off %~
        \begin{lstlisting}[style=normal,firstnumber=1,linebackgroundcolor={%
                 \btLstHLB<1>{3}%
                 \btLstHLB<2>{4,9}%
                 \btLstHLB<3>{11}%
                 \btLstHLB<4>{6}%
                 \btLstHLB<5>{7}%
         }]~% else %~
        \begin{lstlisting}[style=normal,firstnumber=1]~% endif %~
// con referencias
int foo() {
    Vector v(3);
    bar(v);

    v.get(0);
}

void bar(Vector& w) {
  for (int i = 0; /*...*/)
    w.set(i, 1);
}
         \end{lstlisting}

      \end{column}
      \begin{column}{.5\linewidth}
                   ~% if interactive is off %~
         \begin{tikzpicture}[cell/.style={rectangle,draw=black},
            space/.style={minimum height=1.5em,matrix of nodes,row sep=-\pgflinewidth,column sep=-\pgflinewidth,column 1/.style={font=\ttfamily}},text depth=0.5ex,text height=2ex,nodes in empty cells]
            \matrix (stackname)[space, column 1/.style={font=\ttfamily},column 2/.style={font=\ttfamily,nodes={minimum width=5.5em}}] 
            {
                   & {stack}  \\
            };
            \matrix (foo)[below =0.1em of stackname, space, column 1/.style={font=\ttfamily},column 2/.style={nodes={cell,minimum width=5.5em}},column 3/.style={nodes={cell,minimum width=2em}},column 4/.style={nodes={cell,minimum width=2em}},column 5/.style={nodes={cell,minimum width=2em}}] 
            {
                {\visible<1->{v}}   & {\only<1-4>{v.data}\only<5>{XX}}  \\
                                    & {\only<1-4>{v.size = 3}\only<5>{XX}}  \\
            };
             \matrix (bar)[nodevisible on={<2-3>}, below =0.1em of foo, space, column 1/.style={font=\ttfamily},column 2/.style={nodes={cell,minimum width=5.5em}},column 3/.style={nodes={cell,minimum width=2em}},column 4/.style={nodes={cell,minimum width=2em}},column 5/.style={nodes={cell,minimum width=2em}}] 
            {
                {\visible<2->{w}}   & {\visible<2->{\&v}}  \\
            };

            \matrix (heapname)[right=2em of foo, space, column 1/.style={font=\ttfamily},column 2/.style={font=\ttfamily,nodes={minimum width=5.5em}}] 
            {
                   & {heap}  \\
            };
             \matrix (heap)[below =0.1em of heapname, space, column 1/.style={nodes={cell,minimum width=2em}},column 2/.style={nodes={cell,minimum width=2em}},column 3/.style={nodes={cell,minimum width=2em}},column 4/.style={nodes={cell,minimum width=2em}},column 5/.style={nodes={cell,minimum width=2em}}] 
            {
                       {\only<1-2>{0}\only<3-4>{1}\only<5>{XX}} & {\only<1-2>{0}\only<3-4>{1}\only<5>{XX}} & {\only<1-2>{0}\only<3-4>{1}\only<5>{XX}} \\
            };

             \draw [->, pointer, arrowvisible on={<2-3>}] (bar-1-2.west) to[out=115,in=245] (foo-1-2.west);
             \draw [->, pointer] (foo-1-2.east) to (heap.west);


         \end{tikzpicture}
        \begin{onlyenv}<1>
        \begin{lstlisting}[style=normal,firstnumber=5]
  Vector(int size) {  // create
    data = malloc(..);
    memset(data, 0 ..);
    this->size = size;
  }
        \end{lstlisting}
        \end{onlyenv}
        \begin{visibleenv}<5>
        \begin{lstlisting}[style=normal,firstnumber=10]
  ~Vector() { // destroy
    free(data);
  }
        \end{lstlisting}
        \end{visibleenv}
                   ~% endif %~
      \end{column}
   \end{columns}
\end{frame}
\note[itemize] {
\item En C todo se pasa por copia. Si queremos pasar por referencia en realidad se pasa por copia un puntero.
\item En C++ podemos usar el pasaje por referencia. Una referencia es como un alias del objeto referenciado.
}

\begin{frame}[fragile]{Diferencias entre referencias y punteros}{}
   \begin{columns}[T]
      \begin{column}{.4\linewidth}
                   ~% if interactive is off %~
        \begin{lstlisting}[style=normal,firstnumber=1,linebackgroundcolor={%
                 \btLstHLB<1>{1}%
                 \btLstHLB<2>{2}%
                 \btLstHLB<3>{6-7}%
         }]~% else %~
        \begin{lstlisting}[style=normal,firstnumber=1]~% endif %~
int* p = nullptr;
int* q;

int i = 1, j = 2;

int* r = &i;
*r = j;
        \end{lstlisting}
      \end{column}
      \begin{column}{.4\linewidth}
                   ~% if interactive is off %~
        \begin{lstlisting}[style=normal,firstnumber=1,linebackgroundcolor={%
                 \btLstHLR<1>{1}%
                 \btLstHLR<2>{2}%
                 \btLstHLB<3>{6-7}%
         }]~% else %~
        \begin{lstlisting}[style=normal,firstnumber=1]~% endif %~
int& p = nullptr;
int& q;

int i = 1, j = 2;

int& r = i;
r = j;
        \end{lstlisting}
      \end{column}
   \end{columns}
\end{frame}
\note[itemize] {
\item Las referencias en C++ deben ser inicializadas al construirse y una vez que referencian a algun objeto no pueden referenciar a otro.
\item Las referencias funcionan como un alias y el compilador en algunos casos ni siquiera reservara memoria para una referencia.
\item En cambio, los punteros pueden crearse sin inicializar, cambiar de objeto al que apuntan y siempre consumen memoria.
\item Como colorario, las referencias no pueden referenciar a \lstinline[style=normal]!null!s. Una referencia nunca puede ser \lstinline[style=normal]!null!! Es muy \'util y reduce la posibilidad de crashes.
}

\subsection{Pasaje por copia}
\begin{frame}[fragile]{Pasaje por copia naive: bit a bit}
   \begin{columns}[T]
      \begin{column}{.45\linewidth}
                   ~% if interactive is off %~
        \begin{lstlisting}[style=normal,firstnumber=1,linebackgroundcolor={%
                 \btLstHLB<1>{3}%
                 \btLstHLB<2>{4,9}%
                 \btLstHLB<3>{11}%
                 \btLstHLB<4>{12}%
                 \btLstHLR<5>{6}%
                 \btLstHLR<6>{7}%
         }]~% else %~
        \begin{lstlisting}[style=normal,firstnumber=1]~% endif %~
// por copia
int foo() {
    Vector v(3);
    bar(v);

    v.get(0);
}

void bar(Vector w) {
  for (int i = 0; /*...*/)
    w.set(i, 1);
}
         \end{lstlisting}

      \end{column}
      \begin{column}{.5\linewidth}
                   ~% if interactive is off %~
         \begin{tikzpicture}[cell/.style={rectangle,draw=black},
            space/.style={minimum height=1.5em,matrix of nodes,row sep=-\pgflinewidth,column sep=-\pgflinewidth,column 1/.style={font=\ttfamily}},text depth=0.5ex,text height=2ex,nodes in empty cells]
            \matrix (stackname)[space, column 1/.style={font=\ttfamily},column 2/.style={font=\ttfamily,nodes={minimum width=5.5em}}] 
            {
                   & {stack}  \\
            };
            \matrix (foo)[below =0.1em of stackname, space, column 1/.style={font=\ttfamily},column 2/.style={nodes={cell,minimum width=5.5em}},column 3/.style={nodes={cell,minimum width=2em}},column 4/.style={nodes={cell,minimum width=2em}},column 5/.style={nodes={cell,minimum width=2em}}] 
            {
                {\visible<1->{v}}   & {\only<1-5>{v.data}\only<6>{XX}}  \\
                                    & {\only<1-5>{v.size = 3}\only<6>{XX}}  \\
            };
             \matrix (bar)[nodevisible on={<2-4>}, below =0.1em of foo, space, column 1/.style={font=\ttfamily},column 2/.style={nodes={cell,minimum width=5.5em}},column 3/.style={nodes={cell,minimum width=2em}},column 4/.style={nodes={cell,minimum width=2em}},column 5/.style={nodes={cell,minimum width=2em}}] 
            {
                {\visible<1->{w}}   & {\only<1-3>{w.data}\only<4>{XX}}  \\
                                    & {\only<1-3>{w.size = 3}\only<4>{XX}}  \\
            };

            \matrix (heapname)[right=2em of foo, space, column 1/.style={font=\ttfamily},column 2/.style={font=\ttfamily,nodes={minimum width=5.5em}}] 
            {
                   & {heap}  \\
            };
             \matrix (heap)[nodevisible on={<1-4>}, below =0.1em of heapname, space, column 1/.style={nodes={cell,minimum width=2em}},column 2/.style={nodes={cell,minimum width=2em}},column 3/.style={nodes={cell,minimum width=2em}},column 4/.style={nodes={cell,minimum width=2em}},column 5/.style={nodes={cell,minimum width=2em}}] 
            {
                       {\only<1-2>{0}\only<3>{1}\only<4-5>{XX}} & {\only<1-2>{0}\only<3>{1}\only<4-5>{XX}} & {\only<1-2>{0}\only<3>{1}\only<4-5>{XX}} \\
            };

             \draw [->, pointer] (foo-1-2.east) to (heap.west);
             \draw [->, pointer, arrowvisible on={<2-4>}] (bar-1-2.east) to (heap.west);

         \end{tikzpicture}
        \begin{onlyenv}<1>
        \begin{lstlisting}[style=normal,firstnumber=5]
  Vector(int size) {  // create
    data = malloc(..);
    memset(data, 0 ..);
    this->size = size;
  }
        \end{lstlisting}
        \end{onlyenv}
        \begin{onlyenv}<2>
Constructor por copia por default.
        \end{onlyenv}
        \begin{onlyenv}<5>
Use after free!!
        \end{onlyenv}
        \begin{onlyenv}<6>
Double free!!
        \end{onlyenv}
        \begin{visibleenv}<4,6>
        \begin{lstlisting}[style=normal,firstnumber=10]
  ~Vector() { // destroy
    free(data);
  }
        \end{lstlisting}
        \end{visibleenv}
                   ~% endif %~
      \end{column}
   \end{columns}
\end{frame}
\note[itemize] {
\item La copia tanto en C como en C++ es bit a bit y funciona bien para objetos simples.
\item Pero cuando hay punteros, la copia es del puntero y no del valor apuntado: la copia es superficial y no en profundidad (deep copy).
\item Con 2 objetos apuntando al mismo heap, al destruirse uno libera el heap dejando al segundo objeto apuntando a la nada (use after free).
\item Y peor, cuando el segundo objeto se destruya tambi\'en liberar\'a el heap, otra vez (double free).
\item No s\'olo hay problemas con los punteros y el heap, sino tambi\'en con otros tipos de indirecciones como los file descriptors, sockets, threads entre otros.
}

\begin{frame}[fragile]{Constructor por copia}
        \begin{lstlisting}[style=normal,firstnumber=1]
struct Vector {
  int *data;
  int size;

  Vector(const Vector &other) {
    this->data = (int*)malloc(other.size*sizeof(int));
    this->size = other.size;

    memcpy(this->data, other.data, this->size);
  }

};
        \end{lstlisting}
\end{frame}
\note[itemize] {
\item Para crear un objeto nuevo a partir de otro se invoca al constructor por copia.
\item Como cualquier otro constructor, el constructor por copia tiene una member initialization list para pasarle argumentos a los constructores de sus atributos.
\item Todos los objetos en C++ son copiables por default. Si un objeto no tiene un constructor por copia, C++ le crear\'a un constructor por copia por default que implementa una copia bit a bit naive. Por esta razon es muy f\'acil que un objeto se copie sin querer, algo que es dif\'icil de debuggear.
}

\begin{frame}[fragile]{Pasaje por copia: constructor por copia}
   \begin{columns}[T]
      \begin{column}{.45\linewidth}
                   ~% if interactive is off %~
        \begin{lstlisting}[style=normal,firstnumber=1,linebackgroundcolor={%
                 \btLstHLB<1>{3}%
                 \btLstHLB<2>{4,9}%
                 \btLstHLB<3>{12}%
                 \btLstHLB<4>{6}%
                 \btLstHLB<5>{7}%
         }]~% else %~
        \begin{lstlisting}[style=normal,firstnumber=1]~% endif %~
// por copia
int foo() {
    Vector v(3);
    bar(v);

    v.get(0);
}

void bar(Vector w) {
  for (int i = 0; /*...*/)
    w.set(i, 1);
}
         \end{lstlisting}

      \end{column}
      \begin{column}{.5\linewidth}
                   ~% if interactive is off %~
         \begin{tikzpicture}[cell/.style={rectangle,draw=black},
            space/.style={minimum height=1.5em,matrix of nodes,row sep=-\pgflinewidth,column sep=-\pgflinewidth,column 1/.style={font=\ttfamily}},text depth=0.5ex,text height=2ex,nodes in empty cells]
            \matrix (stackname)[space, column 1/.style={font=\ttfamily},column 2/.style={font=\ttfamily,nodes={minimum width=5.5em}}] 
            {
                   & {stack}  \\
            };
            \matrix (foo)[below =0.1em of stackname, space, column 1/.style={font=\ttfamily},column 2/.style={nodes={cell,minimum width=5.5em}},column 3/.style={nodes={cell,minimum width=2em}},column 4/.style={nodes={cell,minimum width=2em}},column 5/.style={nodes={cell,minimum width=2em}}] 
            {
                {\visible<1->{v}}   & {\only<1-4>{v.data}\only<5>{XX}}  \\
                                    & {\only<1-4>{v.size = 3}\only<5>{XX}}  \\
            };
             \matrix (bar)[nodevisible on={<2-3>}, below =0.1em of foo, space, column 1/.style={font=\ttfamily},column 2/.style={nodes={cell,minimum width=5.5em}},column 3/.style={nodes={cell,minimum width=2em}},column 4/.style={nodes={cell,minimum width=2em}},column 5/.style={nodes={cell,minimum width=2em}}] 
            {
                {\visible<1->{w}}   & {\only<1-2>{w.data}\only<3>{XX}}  \\
                                    & {\only<1-2>{w.size = 3}\only<3>{XX}}  \\
            };

            \matrix (heapname)[right=2em of foo, space, column 1/.style={font=\ttfamily},column 2/.style={font=\ttfamily,nodes={minimum width=5.5em}}] 
            {
                   & {heap}  \\
            };
             \matrix (heap)[nodevisible on={<1->}, below =0.1em of heapname, space, column 1/.style={nodes={cell,minimum width=2em}},column 2/.style={nodes={cell,minimum width=2em}},column 3/.style={nodes={cell,minimum width=2em}},column 4/.style={nodes={cell,minimum width=2em}},column 5/.style={nodes={cell,minimum width=2em}}] 
            {
                       {\only<1-4>{0}\only<5>{XX}} & {\only<1-4>{0}\only<5>{XX}} & {\only<1-4>{0}\only<5>{XX}} \\
            };
             \matrix (heap2)[nodevisible on={<2-3>}, below =0.1em of heap, space, column 1/.style={nodes={cell,minimum width=2em}},column 2/.style={nodes={cell,minimum width=2em}},column 3/.style={nodes={cell,minimum width=2em}},column 4/.style={nodes={cell,minimum width=2em}},column 5/.style={nodes={cell,minimum width=2em}}] 
            {
                       {\only<1-2>{0}\only<3-4>{XX}} & {\only<1-2>{0}\only<3-4>{XX}} & {\only<1-2>{0}\only<3-4>{XX}} \\
            };

             \draw [->, pointer] (foo-1-2.east) to (heap.west);
             \draw [->, pointer, arrowvisible on={<2-3>}] (bar-1-2.east) to (heap2.west);

         \end{tikzpicture}
        \begin{onlyenv}<1>
        \begin{lstlisting}[style=normal,firstnumber=5]
  Vector(int size) {  // create
    data = malloc(..);
    memset(data, 0 ..);
    this->size = size;
  }
        \end{lstlisting}
        \end{onlyenv}
        \begin{onlyenv}<2>
        \begin{lstlisting}[style=normal,firstnumber=5]
  Vector(const Vector &other) {
    data = malloc(..);
    size = other.size;

    memcpy(data, other.data, ..);
  }
        \end{lstlisting}
        \end{onlyenv}
        \begin{visibleenv}<3,5>
        \begin{lstlisting}[style=normal,firstnumber=10]
  ~Vector() { // destroy
    free(data);
  }
        \end{lstlisting}
        \end{visibleenv}
                   ~% endif %~
      \end{column}
   \end{columns}
\end{frame}
\note[itemize] {
\item En C y en C++ el pasaje por default es por copia: cuidado de hacer una copia sin intenci\'on, puede traer un comportamiento inesperado (como en el ejemplo) y ser ineficiente.
\item Si no se implementa un constructor por copia se corre el riesgo de caer en un use after free o double free o similar.
\item Evitar a toda costa las copias, son la principal causa de ineficiencias en c\'odigo C y C++.
}

\subsection{Pasaje por movimiento: Move semantics}
\begin{frame}[fragile]{Ownership}
   \begin{columns}[T]
      \begin{column}{.45\linewidth}
          Cada objeto se hacer cargo de sus recursos. Tienen el ownership de ellos.
         \begin{tikzpicture}[cell/.style={rectangle,draw=black},
            space/.style={minimum height=1.5em,matrix of nodes,row sep=-\pgflinewidth,column sep=-\pgflinewidth,column 1/.style={font=\ttfamily}},text depth=0.5ex,text height=2ex,nodes in empty cells]
            \matrix (foo)[space, column 1/.style={font=\ttfamily},column 2/.style={nodes={cell,minimum width=5.5em}},column 3/.style={nodes={cell,minimum width=2em}},column 4/.style={nodes={cell,minimum width=2em}},column 5/.style={nodes={cell,minimum width=2em}}] 
            {
                {v}   & {v.data}  \\
                                    & {v.size = 3}  \\
            };
             \matrix (bar)[below =0.1em of foo, space, column 1/.style={font=\ttfamily},column 2/.style={nodes={cell,minimum width=5.5em}},column 3/.style={nodes={cell,minimum width=2em}},column 4/.style={nodes={cell,minimum width=2em}},column 5/.style={nodes={cell,minimum width=2em}}] 
            {
                {w}   & {w.data}  \\
                                    & {w.size = 3}  \\
            };

            \matrix (heapname)[right=2em of foo, space, column 1/.style={font=\ttfamily},column 2/.style={font=\ttfamily,nodes={minimum width=5.5em}}] 
            {
                   & {heap}  \\
            };
             \matrix (heap)[below =0.1em of heapname, space, column 1/.style={nodes={cell,minimum width=2em}},column 2/.style={nodes={cell,minimum width=2em}},column 3/.style={nodes={cell,minimum width=2em}},column 4/.style={nodes={cell,minimum width=2em}},column 5/.style={nodes={cell,minimum width=2em}}] 
            {
                       {0} & {0} & {0} \\
            };
             \matrix (heap2)[below =0.1em of heap, space, column 1/.style={nodes={cell,minimum width=2em}},column 2/.style={nodes={cell,minimum width=2em}},column 3/.style={nodes={cell,minimum width=2em}},column 4/.style={nodes={cell,minimum width=2em}},column 5/.style={nodes={cell,minimum width=2em}}]
            {
                       {1} & {1} & {1} \\
            };

             \draw [->, pointer] (foo-1-2.east) to (heap.west);
             \draw [->, pointer] (bar-1-2.east) to (heap2.west);

         \end{tikzpicture}

      \end{column}
      \begin{column}{.5\linewidth}
          \pause
          Ambos objetos comparten los recursos: no hay un ownership claro.
         \begin{tikzpicture}[cell/.style={rectangle,draw=black},
            space/.style={minimum height=1.5em,matrix of nodes,row sep=-\pgflinewidth,column sep=-\pgflinewidth,column 1/.style={font=\ttfamily}},text depth=0.5ex,text height=2ex,nodes in empty cells]
            \matrix (foo)[space, column 1/.style={font=\ttfamily},column 2/.style={nodes={cell,minimum width=5.5em}},column 3/.style={nodes={cell,minimum width=2em}},column 4/.style={nodes={cell,minimum width=2em}},column 5/.style={nodes={cell,minimum width=2em}}] 
            {
                {v}   & {v.data}  \\
                                    & {v.size = 3}  \\
            };
             \matrix (bar)[below =0.1em of foo, space, column 1/.style={font=\ttfamily},column 2/.style={nodes={cell,minimum width=5.5em}},column 3/.style={nodes={cell,minimum width=2em}},column 4/.style={nodes={cell,minimum width=2em}},column 5/.style={nodes={cell,minimum width=2em}}] 
            {
                {w}   & {w.data}  \\
                                    & {w.size = 3}  \\
            };

            \matrix (heapname)[right=2em of foo, space, column 1/.style={font=\ttfamily},column 2/.style={font=\ttfamily,nodes={minimum width=5.5em}}] 
            {
                   & {heap}  \\
            };
             \matrix (heap)[below =0.1em of heapname, space, column 1/.style={nodes={cell,minimum width=2em}},column 2/.style={nodes={cell,minimum width=2em}},column 3/.style={nodes={cell,minimum width=2em}},column 4/.style={nodes={cell,minimum width=2em}},column 5/.style={nodes={cell,minimum width=2em}}]
            {
                       {0} & {0} & {0} \\
            };

             \draw [->, pointer] (foo-1-2.east) to (heap.west);
             \draw [->, pointer] (bar-1-2.east) to (heap.west);
         \end{tikzpicture}
      \end{column}
   \end{columns}
\end{frame}

\begin{frame}[fragile]{Constructor por movimiento: transferencia del ownership}
                   ~% if interactive is off %~
        \begin{lstlisting}[style=normal,firstnumber=1,linebackgroundcolor={%
                 \only<1>{\def\lst@linebgrdcmd####1####2####3{}}%
                 \btLstHLB<2>{6,7}%
                 \btLstHLB<3>{9,10}%
                 \btLstHLB<4>{14,15}%
         }]~% else %~
        \begin{lstlisting}[style=normal,firstnumber=1]~% endif %~
struct Vector {
  int *data;
  int size;

  Vector(Vector&& other) {
    this->data = other.data;
    this->size = other.size;

    other.data = nullptr;
    other.size = 0;
  }

  ~Vector() {
      if (data)
          free(data);
  }
};
        \end{lstlisting}
\end{frame}
\note[itemize] {
\item A diferencia de una copia, el constructor por movimiento le roba o mueve los atributos del objeto fuente. 
\item Para marcar el cambio de ownership es necesario modificar al objeto fuente (\lstinline[style=normal]!other!) (por eso no debe ser una constante). Debe dejar de apuntar a los recursos ahora apropiados, de otro modo tendr\'iamos 2 objetos apuntando a un mismo recurso y un bug de memoria a la vuelta de la esquina.
\item Es importante aclarar que luego que el objeto fue movido (\lstinline[style=normal]!other!) debe seguir siendo v\'alido de tal manera que se le puede ejecutar sobre \lstinline[style=normal]!other! el operador asignaci\'on y el destructor. La implementaci\'on de estos dos m\'etodos deben ser acordes como en el ejemplo en donde el destructor pregunta si \lstinline[style=normal]!data == nullptr!
\item C\'omo se implementa la transferencia del ownership depender\'a de cada objeto. En este caso, al poner el puntero \lstinline[style=normal]!data = nullptr! indicamos que no tiene m\'as el ownership del recurso y por lo tanto no tiene que destruirlo.
}

\begin{frame}[fragile]{Pasaje por movimiento}
   \begin{columns}[T]
      \begin{column}{.45\linewidth}
                   ~% if interactive is off %~
        \begin{lstlisting}[style=normal,firstnumber=1,linebackgroundcolor={%
                 \btLstHLB<1>{3}%
                 \btLstHLB<2>{4,9}%
                 \btLstHLB<3>{12}%
                 \btLstHLR<4>{6}%
                 \btLstHLB<5>{7}%
         }]~% else %~
        \begin{lstlisting}[style=normal,firstnumber=1]~% endif %~
// por movimiento
int foo() {
    Vector v(3);
    bar(std::move(v));

    v.get(0); // ??
}

void bar(Vector w) {
  for (int i = 0; /*...*/)
    w.set(i, 1);
}
         \end{lstlisting}

      \end{column}
      \begin{column}{.5\linewidth}
                   ~% if interactive is off %~
         \begin{tikzpicture}[cell/.style={rectangle,draw=black},
            space/.style={minimum height=1.5em,matrix of nodes,row sep=-\pgflinewidth,column sep=-\pgflinewidth,column 1/.style={font=\ttfamily}},text depth=0.5ex,text height=2ex,nodes in empty cells]
            \matrix (stackname)[space, column 1/.style={font=\ttfamily},column 2/.style={font=\ttfamily,nodes={minimum width=5.5em}}] 
            {
                   & {stack}  \\
            };
            \matrix (foo)[below =0.1em of stackname, space, column 1/.style={font=\ttfamily},column 2/.style={nodes={cell,minimum width=5.5em}},column 3/.style={nodes={cell,minimum width=2em}},column 4/.style={nodes={cell,minimum width=2em}},column 5/.style={nodes={cell,minimum width=2em}}] 
            {
                {\visible<1->{v}}   & {\only<1>{v.data}\only<2-4>{v.data = 0}\only<5>{XX}}  \\
                                    & {\only<1>{v.size = 3}\only<2-4>{v.size = 0}\only<5>{XX}}  \\
            };
             \matrix (bar)[nodevisible on={<2-3>}, below =0.1em of foo, space, column 1/.style={font=\ttfamily},column 2/.style={nodes={cell,minimum width=5.5em}},column 3/.style={nodes={cell,minimum width=2em}},column 4/.style={nodes={cell,minimum width=2em}},column 5/.style={nodes={cell,minimum width=2em}}] 
            {
                {\visible<1->{w}}   & {\only<2>{w.data}\only<3>{XX}}  \\
                                    & {\only<2>{w.size = 3}\only<3>{XX}}  \\
            };

            \matrix (heapname)[right=2em of foo, space, column 1/.style={font=\ttfamily},column 2/.style={font=\ttfamily,nodes={minimum width=5.5em}}] 
            {
                   & {heap}  \\
            };
             \matrix (heap)[nodevisible on={<1-3>}, below =0.1em of heapname, space, column 1/.style={nodes={cell,minimum width=2em}},column 2/.style={nodes={cell,minimum width=2em}},column 3/.style={nodes={cell,minimum width=2em}},column 4/.style={nodes={cell,minimum width=2em}},column 5/.style={nodes={cell,minimum width=2em}}]
            {
                       {\only<1-2>{0}\only<3>{XX}} & {\only<1-2>{0}\only<3>{XX}} & {\only<1-2>{0}\only<3>{XX}} \\
            };

             \draw [->, pointer, arrowvisible on={<1>}] (foo-1-2.east) to (heap.west);
             \draw [->, pointer, arrowvisible on={<2-3>}] (bar-1-2.east) to (heap.west);

         \end{tikzpicture}
        \begin{onlyenv}<1>
        \begin{lstlisting}[style=normal,firstnumber=5]
  Vector(int size) {  // create
    data = malloc(..);
    memset(data, 0 ..);
    this->size = size;
  }
        \end{lstlisting}
        \end{onlyenv}
        \begin{onlyenv}<2>
        \begin{lstlisting}[style=normal,firstnumber=5]
  Vector(Vector&& other) {
    this->data = other.data;
    this->size = other.size;

    other.data = nullptr;
    other.size = 0;
  }
        \end{lstlisting}
        \end{onlyenv}
        \begin{onlyenv}<4>
Use after free!! Pero fue mi culpa. Cuando un objeto es movido solo se le puede invocar el operador asignaci\'on o el destructor, cualquier otra cosa esta mal.
        \end{onlyenv}
        \begin{onlyenv}<5>
Pero no hay un double free.
        \end{onlyenv}
        \begin{visibleenv}<3,5>
        \begin{lstlisting}[style=normal,firstnumber=10]
  ~Vector() { // destroy
     if (data)
        free(data);
  }
        \end{lstlisting}
        \end{visibleenv}
                   ~% endif %~
      \end{column}
   \end{columns}
\end{frame}

\begin{frame}[fragile]{Motivaci\'on: retorno de un Socket}{}
Por referencia? \only<2->{Ineficiente o viola RAII}
        \begin{lstlisting}[style=normal,firstnumber=1]
Socket s;
acep.accept(s);
        \end{lstlisting}

        \begin{lstlisting}[style=normal,firstnumber=10]
void accept(Socket &s) {
    close(s.fd);
    s.fd = ::accept(/*...*/); // accept de C
}
        \end{lstlisting}

        \pause
        \pause
Usando el heap? \only<4->{Y si nos olvidamos del \lstinline[style=normal]!delete!?}
        \begin{lstlisting}[style=normal,firstnumber=1]
Socket *s = acep.accept();
        \end{lstlisting}
        \begin{lstlisting}[style=normal,firstnumber=10]
Socket* accept() {
    int fd = ::accept(/*...*/); // accept de C
    return new Socket(fd);
}
        \end{lstlisting}
        \pause
        \pause
Retornar una copia? \only<6->{Tiene sentido? Simplemente No.}

\end{frame}
\note[itemize] {
\item Por referencia: Ineficiente, creamos un \lstinline[style=normal]!fd! para cerrarlo y asignar otro.
\item Por referencia: O bien, el constructor de \lstinline[style=normal]!Socket! no abren ningun \lstinline[style=normal]!fd! (pero entonces no ser\'ian RAII)
\item Por heap: El constructor \lstinline[style=normal]!Socket(int)! debe ser privado.
\item Por heap: Perdemos la ventaja de usar el stack.
}

\begin{frame}[fragile]{Soluci\'on?: Mover el Socket!!}{}
Por movimiento!
        \begin{lstlisting}[style=normal,firstnumber=1]
Socket s = acep.accept();
        \end{lstlisting}
        \begin{lstlisting}[style=normal,firstnumber=10]
Socket accept() {
    int fd = ::accept(/*...*/); // accept de C
    return std::move(Socket(fd));
}
        \end{lstlisting}
   \begin{itemize}
	   \item El constructor \lstinline[style=normal]!Socket(int)! debe ser privado.
	   \item El socket creado dentro del m\'etodo \lstinline[style=normal]!accept! es movido hacia afuera.
	   \item Todos los objetos involucrados viven en el stack y por lo tanto se destruyen autom\'aticamente.
   \end{itemize}
\end{frame}
\note[itemize] {
\item El m\'etodo \lstinline[style=normal]!accept! retorna un nuevo objeto \lstinline[style=normal]!Socket!.
\item El m\'etodo no quiere tener el ownership del nuevo socket creado, quiere moverlo y darselo a quien lo llam\'o.
\item En C y en C++ antes del est\'andar C++11 no hab\'ia otra forma que o pasaje por referencia (lo que implicaba que hab\'ia que construir previamente un \lstinline[style=normal]!Socket! dummy para luego inicializarlo correctamente dentro de \lstinline[style=normal]!accept!) o bien retornarlo usando el heap (perdiendo el beneficio de ser RAII).
\item El compilador puede deducir que el objeto \lstinline[style=normal]!accepted! se lo desea mover. Usar \lstinline[style=normal]!std::move! para hacerlo expl\'icito.
}


\begin{frame}[fragile]{Otro ejemplo: pasando objetos a un hilo}{}
        \begin{lstlisting}[style=normal,firstnumber=10]
std::thread aceptar_un_cliente(Socket &aceptador) {
    Socket skt_cliente = aceptador.accept();

    // movimiento de un socket, todo ok
    std::thread t {manejador_del_cliente,
                   std::move(skt_cliente)};

    return std::move(t);
} // <--el socket skt_cliente se destruye, pero como se movio
  // no deberia pasar nada (siempre que se implemente el
  // constructor por movimiento y el destructor acorde!)

        \end{lstlisting}
\end{frame}
\note[itemize] {
\item La funci\'on crea un nuevo socket \lstinline[style=normal]!skt_cliente! y se lo pasa a un hilo para que lo procese en paralelo.
\item El fin de \lstinline[style=normal]!aceptar_un_cliente! no implica que el socket \lstinline[style=normal]!skt_cliente! se deba cerrar: el lifetime del objeto deber\'ia estar atado al del hilo.
\item No podemos pasar una copia ya que no tiene sentido copiar un socket.
\item Tampoco una referencia ya que el objeto \lstinline[style=normal]!skt_cliente! vive en el stack frame de \lstinline[style=normal]!aceptar_un_cliente! y se destruir\'a al finalizar esta.
\item En C y en C++ antes del est\'andar C++11 no hay otra alternativa que poner el socket \lstinline[style=normal]!skt_cliente! en el heap perdiendo los beneficios RAII. En C++11 se lo mueve directamente.
\item Lo mismo ocurre con el retorno del objeto thread \lstinline[style=normal]!t!.
}


~% endmacro %~

~{ content() }~


\end{document}


