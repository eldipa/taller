
\input{../preamble-tmp.tex}

\title%
{Pasaje de objetos en C++ - Asignaci\'on}

\subject{Pasaje de objetos en C++ - Asignaci\'on}

\tikzset{
  pointer/.style={->, dashed, draw, black, line width=0.03cm},
}

\tikzset{
 nodeinvisible/.style={opacity=.01},
 nodevisible on/.style={alt={#1{}{nodeinvisible}}},
 arrowinvisible/.style={opacity=.01},
 arrowvisible on/.style={alt={#1{}{arrowinvisible}}},
 alt/.code args={<#1>#2#3}{%
   \alt<#1>{\pgfkeysalso{#2}}{\pgfkeysalso{#3}} % \pgfkeysalso doesn't change the path
 },
}

\begin{document}

\begin{frame}[noframenumbering,plain]
   \titlepage
\end{frame}


~% macro content() %~
\section{Asignaci\'on}
\subsection{Asignaci\'on por copia}
\begin{frame}[fragile]{Asignaci\'on}
                   ~% if interactive is off %~
        \begin{lstlisting}[style=normal,firstnumber=1,linebackgroundcolor={%
                 \btLstHLB<1>{1-3,9}%
                 \btLstHLB<2>{7}%
         }]~% else %~
        \begin{lstlisting}[style=normal,firstnumber=1]~% endif %~
Vector f(Vector v) {
  Vector a(v);
  Vector b = v;

  Vector c(5);

  c = v;

  return v;
}
        \end{lstlisting}
\end{frame}
\note[itemize] {
\item En la l\'inea 1 se recibe por copia un vector al que llamaremos \lstinline[style=normal]!v!.
\item En la l\'inea 2 y 3 se crean 2 vectores m\'as copiandose de \lstinline[style=normal]!v!, ambos llaman al constructor por copia.
\item En la l\'inea 9 se retorna un vector por copia tambi\'en salvo que \lstinline[style=normal]!Vector! implemente el constructor por movimiento en cuyo caso \lstinline[style=normal]!v! se mueve y no se copia.
\item En la l\'inea 7 sucede algo distinto. El vector \lstinline[style=normal]!c! copia el contenido del vector \lstinline[style=normal]!v!. Pero el objeto \lstinline[style=normal]!c! ya estaba creado asi que en vez de llamar al constructor por copia llama al operador asignaci\'on por copia.
}

\begin{frame}[fragile]{Asignaci\'on por copia: un objeto creado copiando de otro}
                   ~% if interactive is off %~
        \begin{lstlisting}[style=normal,firstnumber=1,linebackgroundcolor={%
                 \only<1>{\def\lst@linebgrdcmd####1####2####3{}}%
                 \btLstHLB<2>{10,11}%
                 \btLstHLB<3>{13-15}%
                 \btLstHLB<4>{6-8}%
         }]~% else %~
        \begin{lstlisting}[style=normal,firstnumber=1]~% endif %~
struct Vector {
  int *data;
  int size;

  Vector& operator=(const Vector &other) {
    if (this == &other) {
      return *this; // other is myself!
    }

    if (this->data)
        free(this->data);

    this->data = (int*)malloc(other.size*sizeof(int));
    this->size = other.size;
    memcpy(this->data, other.data; this->size);

    return *this;
  }
};
        \end{lstlisting}
\end{frame}
\note[itemize] {
\item Para copiar el contenido de un objeto en otro ya creado se usa el operador asignaci\'on.
\item Como el objeto \lstinline[style=normal]!this! ya esta creado, debemos recordar que todos sus atributos estan ya creados: no podemos cambiar ninguno de sus atributos constantes.
\item Validar que no nos hayan quitado el ownership de nuestros recursos.
\item Todos los objetos en C++ son copiables por asignaci\'on asi que si un objeto no implementa la sobrecarga del operador asignaci\'on, C++ le creara una implementaci\'on por default que hara una copia bit a bit naive.
\item Tambi\'en es posible que nos asignemos a nosotros mismos (haciendo \lstinline[style=normal]!vec = vec;!). Debemos programar el operador asignaci\'on de tal forma que evite copiarse a si mismo.
\item El operador asignaci\'on no es el \'unico operador que se puede sobrecargar. Ya veremos otros y en m\'as detalle en las pr\'oximas clases.
}

\subsection{Asignaci\'on por movimiento}
\begin{frame}[fragile,plain]{Asignaci\'on por movimiento}{}
        \begin{lstlisting}[style=normal,firstnumber=1]
struct Vector {
  int *data;
  int size;

  Vector& operator=(Vector&& other) {
    if (this == &other) {
      return *this; // other is myself!
    }

    if (this->data)
        free(this->data);

    this->data = other.data;
    this->size = other.size;

    other.data = nullptr;
    other.size = 0;

    return *this;
  }
};
        \end{lstlisting}
\end{frame}

\begin{frame}[fragile]{Ej Asignaci\'on por movimiento: swap de objetos}{}
        \begin{lstlisting}[style=normal,firstnumber=10]
void swap(Vector& a, Vector& b) {
    Vector t = a;  // copia (constructor)
    a = b;  // copia (asignacion)
    b = t;  // copia (asignacion)
}
        \end{lstlisting}
        \begin{lstlisting}[style=normal,firstnumber=10]
void swap(Vector& a, Vector& b) {
    Vector t = std::move(a); // a se mueve a t (constructor)
    a = std::move(b); // b se mueve a a (asignacion)
    b = std::move(t); // t se mueve a b (asignacion)
}
        \end{lstlisting}
\end{frame}


%\begin{frame}[plain, fragile]{}{}
%   \begin{columns}[T]
%      \begin{column}{.5\linewidth}
%         \begin{lstlisting}[style=normal,numbers=none]
%void _copy(const Vector &other) {
%  data = malloc(other.size*
%                sizeof(int));
%  size = other.size;
%  memcpy(data, other.data, size);
%}
%
%Vector(const Vector &other) {
%  _copy(other);
%}
%
%Vector&
%operator=(const Vector &other) {
%  if (this == &other)
%    return *this;
%
%  !data || free(data);
%  _copy(other);
%  return *this;
%}
%
%         \end{lstlisting}
%      \end{column}
%      \begin{column}{.45\linewidth}
%         \begin{lstlisting}[style=normal,numbers=none]
%void _move(Vector&& other) {
%  data = other.data;
%  size = other.size;
%  other.data = nullptr;
%  other.size = 0;
%}
%
%Vector(Vector&& other) {
%  _move(other);
%}
%
%Vector&
%operator=(Vector&& other) {
%  if (this == &other)
%    return *this;
%
%  !data || free(data);
%  _move(other);
%  return *this;
%}
%         \end{lstlisting}
%      \end{column}
%   \end{columns}
%\end{frame}

\subsection*{Objetos no copiables}
\begin{frame}[fragile]{ Objetos no copiables}{}
        \begin{lstlisting}[style=normal,firstnumber=1]
struct File {
  public:
  File copy(const char *to_where) { ... }

  private:
  File(const File &other) = delete;
  File& operator=(const File &other) = delete;

};
        \end{lstlisting}
\end{frame}
\note[itemize] {
\item En C++11 podemos decir que tanto el constructor por copia como el de asignaci\'on estan borrados (\lstinline[style=normal]!delete!). Si en algun momento intentamos hacer una copia el compilador dara un error.
\item Pero si trabajamos en C++98, debemos usar algun workaround: declarar y definir el constructor por copia y el operador asignaci\'on y que su implementaci\'on sea fallar (lanzar una excepci\'on). El intento fallido de copia se detecta en runtime.
\item Otra forma seria declarar pero no definir ni el constructor por copia ni el operador asignaci\'on y hacerlos privados. El intento fallido de copia se detecta en tiempo de compilaci\'on y linkeo.
}


~% endmacro %~

~{ content() }~


\end{document}


