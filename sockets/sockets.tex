\pdfminorversion=4

~% if handout is on %~
    \documentclass[professionalfonts,handout]{beamer}
    \setbeameroption{show notes}
    \setbeamertemplate{note page}{\insertnote}
    \setbeamercolor{background canvas}{bg=white}
~% else %~
    ~% if interactive is on %~
        \documentclass[professionalfonts,handout]{beamer}
    ~% else %~
        \documentclass[professionalfonts]{beamer}
    ~% endif %~
~% endif %~



% Template original the Beamer: Copyright 2004 by Till Tantau <tantau@users.sourceforge.net>.


\mode<presentation>
{
  \usetheme{metropolis}
  % or ...

  %\setbeamercovered{transparent} %hace que lo que esta covered se muestre transparente
}

\setbeamertemplate{navigation symbols}{}%remove navigation symbols

\usepackage[english]{babel}

\usepackage[latin1]{inputenc}

\usepackage{times}
\usepackage[T1]{fontenc}

% para poder escribir codigo fuente en las diapositivas
\usepackage{listings}

% para graficar diagramas simples y arboles de jerarquias
\usepackage{tikz}
\usepackage{tikz-qtree}
\usetikzlibrary{matrix,positioning}
\usetikzlibrary{shapes,arrows,chains,calc,decorations.pathmorphing}

\usepackage{xcolor}
\definecolor{darkgreen}{rgb}{0,.5,0}
\definecolor{darkblue}{rgb}{0,0,.7}
\lstdefinestyle{normal}{language=C++,       % lenguaje C++
   numbers=left,                            % enumerar las lineas
   keywordstyle=\color{darkblue}\textbf,    % color de las keywords
   stringstyle=\color{red},                 % color de los strings
   commentstyle=\color{darkgreen},          % color de los comentarios
   basicstyle=\color{black}\ttfamily\footnotesize\bfseries,     % color del texto en general
   morecomment=[l][\color{magenta}]{\#},    % coloreamos las intrucciones del precompilador (todo lo que empieza con #)
   ndkeywords={NULL,nullptr,siz,zer,mov,add,noexcept},               % definimos una nuevas keywords como NULL y nullptr
   ndkeywordstyle=\color{violet},           % y las nuevas keywords tendran este color
   frame=simple,                            % simple, sin ningun marco o frame alrededor del codigo
   basewidth={0.55em,0.55em}                % tamano de las letras/lineas. usado para reducir el whitespace entre estas
}

\lstdefinestyle{normal33}{language=C++,       % lenguaje C++
   numbers=left,                            % enumerar las lineas
   keywordstyle=\color{darkblue}\textbf,    % color de las keywords
   stringstyle=\color{red},                 % color de los strings
   commentstyle=\color{darkgreen},          % color de los comentarios
   basicstyle=\color{black}\ttfamily\footnotesize\bfseries,     % color del texto en general
   morecomment=[l][\color{magenta}]{\#},    % coloreamos las intrucciones del precompilador (todo lo que empieza con #)
   ndkeywords={NULL,nullptr},               % definimos una nuevas keywords como NULL y nullptr
   ndkeywordstyle=\color{violet},           % y las nuevas keywords tendran este color
   frame=simple,                            % simple, sin ningun marco o frame alrededor del codigo
   basewidth={0.55em,0.55em}                % tamano de las letras/lineas. usado para reducir el whitespace entre estas
}
%mismo estilo, sin numeros
\lstdefinestyle{normalnonumbers}{language=C++,
   keywordstyle=\color{darkblue}\textbf,
   stringstyle=\color{red},
   commentstyle=\color{darkgreen},
   basicstyle=\color{black}\ttfamily\footnotesize\bfseries,
   morecomment=[l][\color{magenta}]{\#},
   ndkeywords={NULL,nullptr,move,add},
   ndkeywordstyle=\color{violet},
   frame=simple,
   basewidth={0.55em,0.55em}
}


% mismo estilo pero todos los colores estan mas debiles como si se volvieran transparentes
% usado para poder resaltar el codigo.
\lstdefinestyle{dimmided}{language=C++,
   keywordstyle=\color{darkblue!30}\textbf,
   stringstyle=\color{red!30},
   commentstyle=\color{darkgreen!30},
   basicstyle=\color{black!30}\ttfamily\footnotesize\bfseries,
   morecomment=[l][\color{magenta!30}]{\#},
   ndkeywords={NULL,nullptr,siz,zer,mov,add},
   ndkeywordstyle=\color{violet!30},
   moredelim=**[is][\only<+->{\color{black}\lstset{style=normal}}]{@}{@}, % definimos que las lineas entre arrobas (@) tendran el estilo normal (con los colores a toda intensidad)
   frame=simple,
   basewidth={0.55em,0.55em}
}
\lstdefinestyle{dimmided42}{language=C++,
   keywordstyle=\color{darkblue!30}\textbf,
   stringstyle=\color{red!30},
   commentstyle=\color{darkgreen!30},
   basicstyle=\color{black!30}\ttfamily\footnotesize\bfseries,
   morecomment=[l][\color{magenta!30}]{\#},
   ndkeywords={NULL,nullptr,siz,zer,mov,add},
   ndkeywordstyle=\color{violet!30},
   moredelim=**[is][{\color{black}\lstset{style=normal}}]{@}{@}, % definimos que las lineas entre arrobas (@) tendran el estilo normal (con los colores a toda intensidad)
   frame=simple,
   basewidth={0.55em,0.55em}
}

\lstdefinelanguage{json}{
    literate=
     *{:}{{{\color{violet}{:}}}}{1}
      {,}{{{\color{violet}{,}}}}{1}
      {\{}{{{\color{darkgreen}{\{}}}}{1}
      {\}}{{{\color{darkgreen}{\}}}}}{1}
      {[}{{{\color{darkgreen}{[}}}}{1}
      {]}{{{\color{darkgreen}{]}}}}{1},
}


\lstdefinestyle{normaljson}{language=json,  % lenguaje json
   numbers=left,                            % enumerar las lineas
   keywordstyle=\color{darkblue}\textbf,    % color de las keywords
   stringstyle=\color{red},                 % color de los strings
   commentstyle=\color{red},                % color de los comentarios
   basicstyle=\color{black}\ttfamily\footnotesize\bfseries,     % color del texto en general
   morecomment=[l][\color{magenta}]{\#},    % coloreamos las intrucciones del precompilador (todo lo que empieza con #)
   ndkeywords={NULL,nullptr},               % definimos una nuevas keywords como NULL y nullptr
   ndkeywordstyle=\color{violet},           % y las nuevas keywords tendran este color
   frame=simple,                            % simple, sin ningun marco o frame alrededor del codigo
   basewidth={0.55em,0.55em}                % tamano de las letras/lineas. usado para reducir el whitespace entre estas
}

\lstdefinelanguage{http}{
    morekeywords={
        GET,
        PUT,
        POST,
        DELETE
    }
}

\lstdefinestyle{normalhttp}{language=http,  % lenguaje HTTP
   numbers=left,                            % enumerar las lineas
   keywordstyle=\color{darkblue}\textbf,    % color de las keywords
   stringstyle=\color{red},                 % color de los strings
   commentstyle=\color{red},                % color de los comentarios
   basicstyle=\color{black}\ttfamily\footnotesize\bfseries,     % color del texto en general
   morecomment=[l][\color{magenta}]{\#},    % coloreamos las intrucciones del precompilador (todo lo que empieza con #)
   ndkeywords={NULL,nullptr},               % definimos una nuevas keywords como NULL y nullptr
   ndkeywordstyle=\color{violet},           % y las nuevas keywords tendran este color
   frame=simple,                            % simple, sin ningun marco o frame alrededor del codigo
   basewidth={0.55em,0.55em}                % tamano de las letras/lineas. usado para reducir el whitespace entre estas
}

% Pone las constantes numericas con color (violeta en este caso):
% - los numeros que estan dentro de un string/comentarios NO son coloreados
% - los numeros por fuera que pertenecen a una palabra SON coloreados
%   (buf1, por ejemplo, el "1" es coloreado cuando no lo deberias. UPS!!)
% - No incluye el signo
%
%\lstset{literate=%
%  *{0}{{{\color{violet}0}}}1
%   {1}{{{\color{violet}1}}}1
%   {2}{{{\color{violet}2}}}1
%   {3}{{{\color{violet}3}}}1
%   {4}{{{\color{violet}4}}}1
%   {5}{{{\color{violet}5}}}1
%   {6}{{{\color{violet}6}}}1
%   {7}{{{\color{violet}7}}}1
%   {8}{{{\color{violet}8}}}1
%   {9}{{{\color{violet}9}}}1
%}

% Para resaltar lineas de codigo usando \btLstHLB{range} y \btLstHLB<overlay>{range}:
% Por ejemplo,
%  \btLstHLB{3}       linea 3 resaltada
%  \btLstHLB{1-5}     lineas de la 1 a la 5 resaltadas
%  \btLstHLB<2>{3}    linea 3 resaltada solo en el slide 2
%
% \btLstHLB usa un color azul mientras que \btLstHLR usa un color rojo como fondo
\usepackage{lstlinebgrd}
\makeatletter
\newcount\bt@rangea
\newcount\bt@rangeb

\newcommand\btIfInRange[2]{%
   \global\let\bt@inrange\@secondoftwo%
   \edef\bt@rangelist{#2}%
   \foreach \range in \bt@rangelist {%
      \afterassignment\bt@getrangeb%
      \bt@rangea=0\range\relax%
      \pgfmathtruncatemacro\result{ ( #1 >= \bt@rangea) && (#1 <= \bt@rangeb) }%
      \ifnum\result=1\relax%
      \breakforeach%
      \global\let\bt@inrange\@firstoftwo%
      \fi%
   }%
   \bt@inrange%
}
\newcommand\bt@getrangeb{%
   \@ifnextchar\relax%
   {\bt@rangeb=\bt@rangea}%
   {\@getrangeb}%
}
\def\@getrangeb-#1\relax{%
   \ifx\relax#1\relax%
      \bt@rangeb=100000%   \maxdimen is too large for pgfmath
   \else%
      \bt@rangeb=#1\relax%
   \fi%
}

%%%%%%%%%%%%%%%%%%%%%%%%%%%%%%%%%%%%%%%%%%%%%%%%%%%%%%%%%%%%%%%%%%%%%%%%%%%%%%
%
% \btLstHL<overlay spec>{range list}
%
\newcommand<>{\btLstHLB}[1]{%
   \only#2{\btIfInRange{\value{lstnumber}}{#1}{\color{blue!30}}% blue
   {\def\lst@linebgrdcmd####1####2####3{}}}%
}%
\newcommand<>{\btLstHLR}[1]{%
   \only#2{\btIfInRange{\value{lstnumber}}{#1}{\color{red!30}}% red
   {\def\lst@linebgrdcmd####1####2####3{}}}%
}%
%
%
%%%%%%%%%%%%%%%%%%%%%%

\makeatother


% sin fecha
\date{}

\author[7542-9508]{Di Paola Mart\'in \\ \texttt{martinp.dipaola <at> gmail.com} }

\institute[Universidad de Buenos Aires]
{
   Facultad de Ingenier\'ia\\
   Universidad de Buenos Aires
}

% Definimos una imagen para que este en cada slide
%\pgfdeclareimage[height=0.5cm]{university-logo}{imgs/fiuba.png}
%\logo{\pgfuseimage{university-logo}}

%% Definir estas en el documento final
%%
%% \title%
%% {Programaci\'on gen\'erica y templates en C++}
%%
%% \subject{Programaci\'on gen\'erica y templates en C++}


%%%%%
~% if interactive is on %~
    \AtBeginSection{}
    \AtBeginSubsection{}
~% else %~
    % At begin of each Section do nothing; at each Subsection show the Section and Subsection names
    % If the Section doesn't have a Subsection, then YOU must to add a unnumbered Subsection like \subsection*{}
    % so the AtBeginSubsection will get triggered
    \AtBeginSection{}
    \AtBeginSubsection[\frame{\subsectionpage}]{\frame{\subsectionpage}}
~% endif %~
%
%%%%%




\title%
{Introducci\'on a Sockets TCP en C}


\subject{Introducci\'on a Sockets TCP en C}


\begin{document}

\begin{frame}
   \titlepage
\end{frame}

\begin{frame}{De qu\'e va esto?}
   \tableofcontents
   % You might wish to add the option [pausesections]
\end{frame}

\section{Resoluci\'on de nombres}
\subsection*{}
\begin{frame}{Resoluci\'on de nombres: desde donde quiero escuchar}
    \begin{tikzpicture}[remember picture,overlay]
        \node[yshift=-1cm,at=(current page.center)] {
            \includegraphics<1>[width=\paperwidth]{overview_slides/pg_0002.pdf}
        };
    \end{tikzpicture}
 \end{frame}
\note[itemize] {
\item El servidor tiene que definir desde donde quiere recibir las conexiones.
\item Hay m\'as esquemas posibles pero solo nos interesa definir la IP y el puerto del servidor.
\item Sin embargo, hardcodear la IP y/o el puerto es una mala pr\'actica. Mejor es usar nombres simb\'olicos: host name y service name.
\item La funci\'on \lstinline[style=normal]!getaddrinfo! se encargara de resolver esos nombres y llevarlos a IPs y puertos.
}

\begin{frame}{Resoluci\'on de nombres: a quien me quiero conectar}
    \begin{tikzpicture}[remember picture,overlay]
        \node[yshift=-1cm,at=(current page.center)] {
            \includegraphics<1>[width=\paperwidth]{overview_slides/pg_0003.pdf}
        };
    \end{tikzpicture}
\end{frame}

\begin{frame}{Familias y tipos de sockets}
   \begin{itemize}
     \item<1-> Familia \lstinline[style=normal]!AF_UNIX!: para la comunicaci\'on entre procesos locales.
     \item<2-> Familias \lstinline[style=normal]!AF_INET! (IPv4) y \lstinline[style=normal]!AF_INET6! (IPv6): para la comunicaci\'on a traves de la Internet.
    \end{itemize}
    \begin{itemize}
        \item<3-> Tipo \lstinline[style=normal]!SOCK_DGRAM! (UDP): Sin conexi\'on. Orientado a mensajes (datagramas). Los mensajes se pierden, duplican y llegan en desorden.
        \item<4-> Tipo \lstinline[style=normal]!SOCK_STREAM! (TCP): Con conexi\'on, full-duplex. Orientado al streaming. Los bytes llegan en orden y sin p\'erdidas. \alert{An\'alogo a un archivo binario secuencial.}
    \end{itemize}
\end{frame}

\begin{frame}[fragile]{Resoluci\'on de nombres}
    Cliente
      \begin{lstlisting}[style=normal]
memset(&hints, 0, sizeof(struct addrinfo));
hints.ai_family   = AF_INET;      /* IPv4  */
hints.ai_socktype = SOCK_STREAM;  /* TCP   */
hints.ai_flags    = 0;

status = getaddrinfo("fi.uba.ar", "http", &hints, &results);
      \end{lstlisting}
    Servidor
      \begin{lstlisting}[style=normal]
memset(&hints, 0, sizeof(struct addrinfo));
hints.ai_family   = AF_INET;      /* IPv4  */
hints.ai_socktype = SOCK_STREAM;  /* TCP   */
hints.ai_flags    = AI_PASSIVE;

status = getaddrinfo(0 /* ANY */, "http", &hints, &results);
      \end{lstlisting}
\end{frame}

%\begin{frame}[fragile]{Resoluci\'on de nombres}{No hardcodeen las IPs ni los puertos}
%      \begin{lstlisting}[style=normal]
%#define _POSIX_C_SOURCE 1
%
%struct addrinfo hints, *results, *addr;
%
%int status = getaddrinfo(
%                hostname,
%                service,
%                &hints,   // <- el filtro
%                &results  // <- lista de direcciones
%            );
%/* ... */
%
%freeaddrinfo(results);
%
%      \end{lstlisting}
%\end{frame}

\section{Canal de comunicaci\'on TCP}
\subsection{Establecimiento de un canal}
\begin{frame}{Creaci\'on de un socket}
    \begin{tikzpicture}[remember picture,overlay]
        \node[yshift=-1cm,at=(current page.center)] {
            \includegraphics<1>[width=\paperwidth]{overview_slides/pg_0004.pdf}
        };
    \end{tikzpicture}
\end{frame}
\note[itemize] {
\item Crear un socket no es nada mas que crear un file descriptor al igual que cuando abrimos un archivo.
}


\begin{frame}{Enlazado de un socket a una direcci\'on}
    \begin{tikzpicture}[remember picture,overlay]
        \node[yshift=-1cm,at=(current page.center)] {
            \includegraphics<1>[width=\paperwidth]{overview_slides/pg_0005.pdf}
        };
    \end{tikzpicture}
\end{frame}
\note[itemize] {
\item A los sockets se los puede enlazar o atar a una direcci\'on IP y puerto local para que el sistema operativo sepa desde donde puede enviar y recibir conexiones y mensajes.
\item El uso mas t\'ipico de \lstinline[style=normal]!bind! se da del lado del servidor cuando este dice "quiero escuchar conexiones desde mi IP p\'ublica y en este puerto".
\item Sin embargo el cliente tambi\'en puede hacer \lstinline[style=normal]!bind! por razones un poco mas esot\'ericas.
}

\begin{frame}{Socket aceptador o pasivo}
    \begin{tikzpicture}[remember picture,overlay]
        \node[yshift=-1cm,at=(current page.center)] {
            \includegraphics<1>[width=\paperwidth]{overview_slides/pg_0006.pdf}
            \includegraphics<2>[width=\paperwidth]{overview_slides/pg_0007.pdf}
        };
    \end{tikzpicture}
\end{frame}
\note[itemize] {
\item Una vez enlazado le decimos al sistema operativo que queremos escuchar conexiones en esa IP/puerto.
\item La funci\'on \lstinline[style=normal]!listen! define hasta cuantas conexiones en "espera de ser aceptadas" el sistema operativo puede guardar.
\item La funci\'on \lstinline[style=normal]!listen! NO define un l\'imite de las conexiones totales (en espera + las que estan ya aceptadas). No confundir!
\item Ahora el servidor puede esperar a que alguien quiera conectarse y aceptar la conexi\'on con la funci\'on \lstinline[style=normal]!accept!.
\item La funci\'on \lstinline[style=normal]!accept! es bloqueante.
}

\begin{frame}{Conexi\'on con el servidor: estableciendo conexi\'on}
    \begin{tikzpicture}[remember picture,overlay]
        \node[yshift=-1cm,at=(current page.center)] {
            \includegraphics<1>[width=\paperwidth]{overview_slides/pg_0008.pdf}
        };
    \end{tikzpicture}
\end{frame}
\note[itemize] {
\item El cliente usa su socket para conectarse al servidor. La operaci\'on \lstinline[style=normal]!connect! es bloqueante.
}

\begin{frame}{Conexi\'on con el servidor: aceptando la conexi\'on}
    \begin{tikzpicture}[remember picture,overlay]
        \node[yshift=-1cm,at=(current page.center)] {
            \includegraphics<1>[width=\paperwidth]{overview_slides/pg_0009.pdf}
        };
    \end{tikzpicture}
\end{frame}
\note[itemize] {
\item La conexi\'on es aceptada por el servidor: la funci\'on \lstinline[style=normal]!accept! se desbloquea y retorna un nuevo socket que representa a la nueva conexi\'on.
}

\subsection{Envio y recepci\'on de datos}
\begin{frame}{Conexi\'on establecida}
    \begin{tikzpicture}[remember picture,overlay]
        \node[yshift=-1cm,at=(current page.center)] {
            \includegraphics<1>[width=\paperwidth]{overview_slides/pg_0010.pdf}
        };
    \end{tikzpicture}
\end{frame}
\note[itemize] {
\item El socket \lstinline[style=normal]!acep! sigue estando disponible para que el servidor acepte a otras conexiones en paralelo mientras antiende a sus clientes (es independiente del socket \lstinline[style=normal]!srv!)
\item Al mismo tiempo, el socket \lstinline[style=normal]!srv! quedo asociado a esa conexi\'on en particular y le permitir\'a al servidor enviar y recibir mensajes de su cliente.
\item Tanto el cliente como el servidor se pueden enviar y recibir mensajes (\lstinline[style=normal]!send!/\lstinline[style=normal]!recv!) entre ellos.
\item Los mensajes/bytes enviados con \lstinline[style=normal]!cli.send! son recibidos por el servidor con \lstinline[style=normal]!srv.recv!.
\item De igual modo el cliente recibe con \lstinline[style=normal]!cli.recv! los bytes enviados por el servidor con \lstinline[style=normal]!srv.send!.
}

\begin{frame}{Envio y recepci\'on de datos}
    \begin{tikzpicture}[remember picture,overlay]
        \node[yshift=-1cm,at=(current page.center)] {
            \includegraphics<1>[width=\paperwidth]{overview_slides/pg_0011.pdf}
        };
    \end{tikzpicture}
\end{frame}
\note[itemize] {
\item El par \lstinline[style=normal]!cli.send!--\lstinline[style=normal]!srv.recv! forma un canal en una direcci\'on mientras que el par \lstinline[style=normal]!srv.send!--\lstinline[style=normal]!cli-recv! forma otro canal en el sentido opuesto.
\item Ambos canales son independientes. Esto se lo conoce como comunicaci\'on Full Duplex
\item TCP garantiza que los bytes enviados llegaran en el mismo orden, sin repeticiones y sin p\'erdidas del otro lado.
\item Otro protocolos como UDP no son tan robustos...
}
\begin{frame}{Envio y recepci\'on de datos en la realidad}
    \begin{tikzpicture}[remember picture,overlay]
        \node[yshift=-1cm,at=(current page.center)] {
            \includegraphics<1>[width=\paperwidth]{overview_slides/pg_0012.pdf}
        };
    \end{tikzpicture}
\end{frame}
\begin{frame}{Envio y recepci\'on de datos en la realidad}
    \begin{tikzpicture}[remember picture,overlay]
        \node[yshift=-1cm,at=(current page.center)] {
            \includegraphics<1>[width=\paperwidth]{overview_slides/pg_0020.pdf}
        };
    \end{tikzpicture}
\end{frame}
\note[itemize] {
\item Sin embargo TCP NO garantiza que todos los bytes pasados a \lstinline[style=normal]!send! se puedan enviar en un solo intento: el programador debera hacer m\'ultiples llamadas a \lstinline[style=normal]!send!.
\item De igual modo, no todo lo enviado sera recibido en una \'unica llamada a \lstinline[style=normal]!recv!: el programador debera hacer m\'ultiples llamadas a \lstinline[style=normal]!recv!.
}

\begin{frame}[fragile]{Envio y recepci\'on de datos}
      \begin{lstlisting}[style=normal]
int s = send(skt,
              buf,
              bytes_to_sent,
              flags           // MSG_NOSIGNAL
            );

int s = recv(skt,
              buf,
              bytes_to_recv,
              flags           // 0
            );

  (s == -1)  // Error inesperado, ver errno
   (s == 0)  // El socket fue cerrado
    (s > 0)  // Ok: s bytes fueron enviados/recibidos
      \end{lstlisting}
\end{frame}


\begin{frame}[fragile]{recvall: recepci\'on de N bytes exactos}
      \begin{lstlisting}[style=normal]
char buf[MSG_LEN];  // buffer donde guardar los datos
int bytes_recv = 0;

while (MSG_LEN > bytes_recv && skt_still_open) {
  s = recv(skt, &buf[bytes_recv], MSG_LEN - bytes_recv - 1,
                                                0);
  if (s == -1) { // Error inesperado, ver errno
     /* ... */
  }
  else if (s == 0) { // Nos cerraron el socket
     /* ... */
  }
  else {
    bytes_recv += s;
  }
}
      \end{lstlisting}
\end{frame}
\begin{frame}[fragile]{sendall: env\'io de N bytes exactos}
      \begin{lstlisting}[style=normal]
char buf[MSG_LEN];    // buffer con los datos a enviar
int bytes_sent = 0;

while (MSG_LEN > bytes_sent && skt_still_open) {
  s = send(skt, &buf[bytes_sent], MSG_LEN - bytes_sent,
                                              MSG_NOSIGNAL);
  if (s == -1) { // Error inesperado, ver errno
     /* ... */
  }
  else if (s == 0) { // Nos cerraron el socket
     /* ... */
  }
  else {
    bytes_sent += s;
  }
}
      \end{lstlisting}
\end{frame}

\subsection{Finalizaci\'on de un canal}
\begin{frame}{Cierre de conexi\'on parcial}
    \begin{tikzpicture}[remember picture,overlay]
        \node[yshift=-1cm,at=(current page.center)] {
            \includegraphics<1>[width=\paperwidth]{overview_slides/pg_0013.pdf}
        };
    \end{tikzpicture}
\end{frame}

\begin{frame}{Cierre de conexi\'on total}
    \begin{tikzpicture}[remember picture,overlay]
        \node[yshift=-1cm,at=(current page.center)] {
            \includegraphics<1>[width=\paperwidth]{overview_slides/pg_0014.pdf}
        };
    \end{tikzpicture}
\end{frame}
\note[itemize] {
\item Parcial en un sentido (envio) \lstinline[style=normal]!SHUT_WR!
\item Parcial en el otro sentido (recepci\'on) \lstinline[style=normal]!SHUT_RD!
\item Total en ambos sentidos \lstinline[style=normal]!SHUT_RDWR!
}

\begin{frame}{Liberaci\'on de los recursos con close}
    \begin{tikzpicture}[remember picture,overlay]
        \node[yshift=-1cm,at=(current page.center)] {
            \includegraphics<1>[width=\paperwidth]{overview_slides/pg_0015.pdf}
        };
    \end{tikzpicture}
\end{frame}

\begin{frame}{Cierre y liberaci\'on del socket aceptador}
    \begin{tikzpicture}[remember picture,overlay]
        \node[yshift=-1cm,at=(current page.center)] {
            \includegraphics<1>[width=\paperwidth]{overview_slides/pg_0016.pdf}
        };
    \end{tikzpicture}
\end{frame}

\begin{frame}{TIME WAIT}
    \begin{tikzpicture}[remember picture,overlay]
        \node[yshift=-1cm,at=(current page.center)] {
            \includegraphics<1>[width=\paperwidth]{overview_slides/pg_0017.pdf}
        };
    \end{tikzpicture}
\end{frame}

\begin{frame}[fragile]{TIME WAIT -> Reuse Address}
Si el puerto 80 esta en el estado TIME WAIT, esto termina en error (Address Already in Use):
      \begin{lstlisting}[style=normal]
int acep   = socket(...);
int status = bind(acep, ...); //bind al puerto 80
      \end{lstlisting}
\pause
La soluci\'on es configurar al socket aceptador para que pueda reusar la direcci\'on:
      \begin{lstlisting}[style=normal]
int acep   = socket(...);

int val = 1;
setsockopt(acep, SOL_SOCKET, SO_REUSEADDR, &val, sizeof(val));

int status = bind(acep, ...); //bind al puerto 80
      \end{lstlisting}
\end{frame}

\section{Protocolos y formatos}
\subsection*{}
\begin{frame}[fragile]{Binario o Texto}
    \begin{itemize}
        \item Protocolos en Binario: son simples y eficientes en terminos de memoria y velocidad de procesamiento. Son m\'as dif\'iciles de debuggear. Es necesario tomar en consideraci\'on el endianess, el padding, los tama\~nos y los signos.
        \item Protocolos en Texto: son la contracara de los protocolos binarios, son lentos, ineficientes y m\'as dif\'iciles de parsear pero m\'as f\'aciles de debuggear. Son independientes del endianess, padding y otros pero dependen del encoding del texto y que caracteres se usan como delimitadores.
    \end{itemize}
\end{frame}

\begin{frame}[fragile]{Longitud variable: delimitador}
    Delimitador: el mensaje no tiene un tama\~no fijo y el fin del mensaje esta marcado por un delimitador.
    \begin{lstlisting}[style=normalhttp]
GET /index.html HTTP/1.1\r\n
Host: www.fi.uba.ar\r\n
\r\n
    \end{lstlisting}
    \begin{itemize}
        \item En HTTP el fin del mensaje esta dado por una l\'inea vacia; cada l\'inea esta delimitada por un \lstinline[style=normalhttp]!\r\n!
        \item Cuantos bytes reservar\'ian para contener dicho mensaje o alguna l\'inea?
        \item Que pasa si el delimitador \lstinline[style=normalhttp]!\r\n! aparece en el medio de una l\'inea, como lo diferenciar\'ian?
    \end{itemize}
\end{frame}
\note[itemize] {
\item Habitualmente en protocolos en texto se usa uno o una secuencia de caracteres como delimitadores.
\item En la cabezera de HTTP se usa \lstinline[style=normal]!\\r\\n!
\item En C/C++, los fin de strings son marcados con \lstinline[style=normal]!\\0!
\item Aunque simple, no es trivial saber cuantos bytes hay hasta el delimitador.
\item Tampoco es trivial el caso de que el texto contenga al delimitador meramente por que es parte de su contenido.
\item Hay dos opciones, o se opta por otro protocolo o se usa una secuencia de escape para que el delimitador sea considerado un literal y no un delimitador.
\item Y si la secuencia de escape es parte del contenido? Hay que escapear la secuencia de escape con otra secuencia de escape.
\item Por ejemplo, el compilador de C/C++ ve \lstinline[style=normal]!"\\1"! como un string con el byte 1 (a pesar de haber 2 caracteres). Si se quisiera literalmente poner una barra y un 1 hay que escapear la barra: \lstinline[style=normal]!"\\\\1"!
}

\begin{frame}[fragile]{Longitud variable: prefijo con la longitud}
    \begin{lstlisting}[style=normal]
struct Msj {
    unsigned short type;
    unsigned short length;
    char* value;
};

read(fd, &msj.type, sizeof(unsigned short) * 2);
msj.value = (char*) malloc(msj.length);
read(fd, msj.value, msj.length);
    \end{lstlisting}
    \begin{itemize}
        \item Los primeros 4 bytes indican la longitud y tipo del valor; el resto de los bytes son el valor en s\'i.
        \item Por qu\'e es importante usar \lstinline[style=normal]!unsigned short! y no solamente \lstinline[style=normal]!short!? Qu\'e pasa si \lstinline[style=normal]!sizeof(unsigned short)! no es 2?
        \item Que pasa si el endianess no coincide? y si hay padding entre los dos primeros campos?
    \end{itemize}
\end{frame}
\note[itemize] {
\item Prefijar la longitud del mensaje soluciona varios problemas pero trae otros.
\item Si \lstinline[style=normal]!sizeof(unsigned short)! vale 4 estar\'iamos enviando 8 bytes con las longitud y tipo pero la m\'aquina que recibe el mensaje puede esperar 4.
\item Hay que definir y forzar un endianess y reglas de padding.
}

\section{Netstat}
\subsection*{}
\begin{frame}[fragile,label=netstat]{Netstat}
      \begin{lstlisting}[style=normal]
machineA$ nc -l 1234 &
machineA$ nc -l 8080 &
machineA$ nc 127.0.0.1 8080 &

machineA$ netstat -tauon
Active Internet connections (servers and established)
Proto Local Address       Foreign Address      State
tcp   127.0.0.1:1234      0.0.0.0:*            LISTEN
tcp   127.0.0.1:8080      127.0.0.1:33036      ESTABLISHED
tcp   127.0.0.1:33036     127.0.0.1:8080       ESTABLISHED
      \end{lstlisting}
\end{frame}

\begin{frame}[fragile,label=netstat2]{Netstat}
      \begin{lstlisting}[style=normal]
machineA$ sudo killall -9 nc

machineA$ netstat -tauon
Active Internet connections (servers and established)
Proto Local Address       Foreign Address      State
tcp   127.0.0.1:8080      127.0.0.1:33036      TIME_WAIT

      \end{lstlisting}
\end{frame}


\appendix
\section<presentation>*{\appendixname}
\subsection<presentation>*{Referencias}

\begin{frame}[allowframebreaks]
   \frametitle<presentation>{Referencias}

   \begin{thebibliography}{10}

         \beamertemplatearticlebibitems
         % Followed by interesting articles. Keep the list short.

      \bibitem{manpages}
         man getaddrinfo
      \bibitem{manpages}
         man netcat
      \bibitem{manpages}
         man netstat



   \end{thebibliography}
\end{frame}

\end{document}


