
%diferencia entre texto y binario
%bin / text
%prefix / delim
\input{../preamble-tmp.tex}

\title%
{Sockets TCP/IP en C - Protocolos}


\subject{Sockets TCP/IP en C - Protocolos}


\begin{document}

\begin{frame}[noframenumbering,plain]
   \titlepage
\end{frame}

~% if interactive is off %~
\begin{frame}{De qu\'e va esto?}
   \tableofcontents
   % You might wish to add the option [pausesections]
\end{frame}
~% endif %~


~% macro content() %~
\section{Protocolos y formatos}
\subsection*{}
\begin{frame}[fragile]{Binario o Texto}
    \begin{itemize}
        \item Protocolos en Texto: son la contracara de los protocolos binarios, son lentos, ineficientes y m\'as dif\'iciles de parsear pero m\'as f\'aciles de debuggear. Son independientes del endianess, padding y otros pero dependen del encoding del texto y que caracteres se usan como delimitadores.
        \item Protocolos en Binario: son simples y eficientes en terminos de memoria y velocidad de procesamiento. Son m\'as dif\'iciles de debuggear. Es necesario tomar en consideraci\'on el endianess, el padding, los tama\~nos y los signos.
    \end{itemize}
\end{frame}

\begin{frame}[fragile]{HTTP}
    \begin{lstlisting}[style=normalhttp]
GET /index.html HTTP/1.1\r\n
Host: www.fi.uba.ar\r\n
\r\n
    \end{lstlisting}
~% if interactive is off %~
    \begin{itemize}
        \item En HTTP el fin del mensaje esta dado por una l\'inea vacia; cada l\'inea esta delimitada por un \lstinline[style=normalhttp]!\r\n!
        \item Cuantos bytes reservar\'ian para contener dicho mensaje o alguna l\'inea?
        \item Que pasa si el delimitador \lstinline[style=normalhttp]!\r\n! aparece en el medio de una l\'inea, como lo diferenciar\'ian?
    \end{itemize}
~% endif %~
\end{frame}
\note[itemize] {
\item Habitualmente en protocolos en texto se usa uno o una secuencia de caracteres como delimitadores.
\item En la cabezera de HTTP se usa \lstinline[style=normal]!\\r\\n!
\item En C/C++, los fin de strings son marcados con \lstinline[style=normal]!\\0!
\item Aunque simple, no es trivial saber cuantos bytes hay hasta el delimitador.
\item Tampoco es trivial el caso de que el texto contenga al delimitador meramente por que es parte de su contenido.
\item Hay dos opciones, o se opta por otro protocolo o se usa una secuencia de escape para que el delimitador sea considerado un literal y no un delimitador.
\item Y si la secuencia de escape es parte del contenido? Hay que escapear la secuencia de escape con otra secuencia de escape.
\item Por ejemplo, el compilador de C/C++ ve \lstinline[style=normal]!"\\1"! como un string con el byte 1 (a pesar de haber 2 caracteres). Si se quisiera literalmente poner una barra y un 1 hay que escapear la barra: \lstinline[style=normal]!"\\\\1"!
}

\begin{frame}[fragile]{TLV}
    \begin{lstlisting}[style=normal]
struct Msj {
    unsigned short type;
    unsigned short length;
    char* value;
};

read(fd, &msj.type, sizeof(unsigned short) * 2);
msj.value = (char*) malloc(msj.length);
read(fd, msj.value, msj.length);
    \end{lstlisting}
~% if interactive is off %~
    \begin{itemize}
        \item Los primeros 4 bytes indican la longitud y tipo del valor; el resto de los bytes son el valor en s\'i.
        \item Por qu\'e es importante usar \lstinline[style=normal]!unsigned short! y no solamente \lstinline[style=normal]!short!? Qu\'e pasa si \lstinline[style=normal]!sizeof(unsigned short)! no es 2?
        \item Que pasa si el endianess no coincide? y si hay padding entre los dos primeros campos?
    \end{itemize}
~% endif %~
\end{frame}
\note[itemize] {
\item Prefijar la longitud del mensaje soluciona varios problemas pero trae otros.
\item Si \lstinline[style=normal]!sizeof(unsigned short)! vale 4 estar\'iamos enviando 8 bytes con las longitud y tipo pero la m\'aquina que recibe el mensaje puede esperar 4.
\item Hay que definir y forzar un endianess y reglas de padding.
}


~% endmacro %~

~{ content() }~

\end{document}
