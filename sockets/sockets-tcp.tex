\input{../preamble-tmp.tex}

\title%
{Sockets TCP/IP en C - Sockets}


\subject{Sockets TCP/IP en C - Sockets}


\begin{document}

\begin{frame}[noframenumbering,plain]
   \titlepage
\end{frame}

~% if interactive is off %~
\begin{frame}{De qu\'e va esto?}
   \tableofcontents
   % You might wish to add the option [pausesections]
\end{frame}
~% endif %~


~% macro content() %~

~% if interactive is on %~
\section{}
\begin{frame}{Canal de comunicaci\'on TCP}
\end{frame}
~% endif %~


~% if interactive is off %~
\section{Resoluci\'on de nombres}
\subsection*{}
\begin{frame}{Resoluci\'on de nombres: desde donde quiero escuchar}
    \begin{tikzpicture}[remember picture,overlay]
        \node[yshift=-1cm,at=(current page.center)] {
            \includegraphics<1>[width=\paperwidth]{overview_slides/pg_0002.pdf}
        };
    \end{tikzpicture}
 \end{frame}
\note[itemize] {
\item El servidor tiene que definir desde donde quiere recibir las conexiones.
\item Hay m\'as esquemas posibles pero solo nos interesa definir la IP y el puerto del servidor.
\item Sin embargo, hardcodear la IP y/o el puerto es una mala pr\'actica. Mejor es usar nombres simb\'olicos: host name y service name.
\item La funci\'on \lstinline[style=normal]!getaddrinfo! se encargara de resolver esos nombres y llevarlos a IPs y puertos.
}

\begin{frame}{Resoluci\'on de nombres: a quien me quiero conectar}
    \begin{tikzpicture}[remember picture,overlay]
        \node[yshift=-1cm,at=(current page.center)] {
            \includegraphics<1>[width=\paperwidth]{overview_slides/pg_0003.pdf}
        };
    \end{tikzpicture}
\end{frame}
~% endif %~


\begin{frame}[fragile]{Resoluci\'on de nombres}
    Cliente
      \begin{lstlisting}[style=normal]
memset(&hints, 0, sizeof(struct addrinfo));
hints.ai_family   = AF_INET;      /* IPv4  */
hints.ai_socktype = SOCK_STREAM;  /* TCP   */
hints.ai_flags    = 0;

status = getaddrinfo("fi.uba.ar", "http", &hints, &results);
      \end{lstlisting}
    Servidor
      \begin{lstlisting}[style=normal]
memset(&hints, 0, sizeof(struct addrinfo));
hints.ai_family   = AF_INET;      /* IPv4  */
hints.ai_socktype = SOCK_STREAM;  /* TCP   */
hints.ai_flags    = AI_PASSIVE;

status = getaddrinfo(0 /* ANY */, "http", &hints, &results);
      \end{lstlisting}
      \begin{lstlisting}[style=normal]
freeaddrinfo(results);  // Cliente y Servidor
      \end{lstlisting}
\end{frame}


~% if interactive is off %~
\section{Canal de comunicaci\'on TCP}
\subsection{Establecimiento de un canal}
\begin{frame}{Creaci\'on de un socket}
    \begin{tikzpicture}[remember picture,overlay]
        \node[yshift=-1cm,at=(current page.center)] {
            \includegraphics<1>[width=\paperwidth]{overview_slides/pg_0004.pdf}
        };
    \end{tikzpicture}
\end{frame}
\note[itemize] {
\item Crear un socket no es nada mas que crear un file descriptor al igual que cuando abrimos un archivo.
}


\begin{frame}{Enlazado de un socket a una direcci\'on}
    \begin{tikzpicture}[remember picture,overlay]
        \node[yshift=-1cm,at=(current page.center)] {
            \includegraphics<1>[width=\paperwidth]{overview_slides/pg_0005.pdf}
        };
    \end{tikzpicture}
\end{frame}
\note[itemize] {
\item A los sockets se los puede enlazar o atar a una direcci\'on IP y puerto local para que el sistema operativo sepa desde donde puede enviar y recibir conexiones y mensajes.
\item El uso mas t\'ipico de \lstinline[style=normal]!bind! se da del lado del servidor cuando este dice "quiero escuchar conexiones desde mi IP p\'ublica y en este puerto".
\item Sin embargo el cliente tambi\'en puede hacer \lstinline[style=normal]!bind! por razones un poco mas esot\'ericas.
}

\begin{frame}{Socket aceptador o pasivo}
    \begin{tikzpicture}[remember picture,overlay]
        \node[yshift=-1cm,at=(current page.center)] {
            \includegraphics<1>[width=\paperwidth]{overview_slides/pg_0006.pdf}
            \includegraphics<2>[width=\paperwidth]{overview_slides/pg_0007.pdf}
        };
    \end{tikzpicture}
\end{frame}
\note[itemize] {
\item Una vez enlazado le decimos al sistema operativo que queremos escuchar conexiones en esa IP/puerto.
\item La funci\'on \lstinline[style=normal]!listen! define hasta cuantas conexiones en "espera de ser aceptadas" el sistema operativo puede guardar.
\item La funci\'on \lstinline[style=normal]!listen! NO define un l\'imite de las conexiones totales (en espera + las que estan ya aceptadas). No confundir!
\item Ahora el servidor puede esperar a que alguien quiera conectarse y aceptar la conexi\'on con la funci\'on \lstinline[style=normal]!accept!.
\item La funci\'on \lstinline[style=normal]!accept! es bloqueante.
}

\begin{frame}{Conexi\'on con el servidor: estableciendo conexi\'on}
    \begin{tikzpicture}[remember picture,overlay]
        \node[yshift=-1cm,at=(current page.center)] {
            \includegraphics<1>[width=\paperwidth]{overview_slides/pg_0008.pdf}
        };
    \end{tikzpicture}
\end{frame}
\note[itemize] {
\item El cliente usa su socket para conectarse al servidor. La operaci\'on \lstinline[style=normal]!connect! es bloqueante.
}

\begin{frame}{Conexi\'on con el servidor: aceptando la conexi\'on}
    \begin{tikzpicture}[remember picture,overlay]
        \node[yshift=-1cm,at=(current page.center)] {
            \includegraphics<1>[width=\paperwidth]{overview_slides/pg_0009.pdf}
        };
    \end{tikzpicture}
\end{frame}
\note[itemize] {
\item La conexi\'on es aceptada por el servidor: la funci\'on \lstinline[style=normal]!accept! se desbloquea y retorna un nuevo socket que representa a la nueva conexi\'on.
}

\subsection{Envio y recepci\'on de datos}
\begin{frame}{Conexi\'on establecida}
    \begin{tikzpicture}[remember picture,overlay]
        \node[yshift=-1cm,at=(current page.center)] {
            \includegraphics<1>[width=\paperwidth]{overview_slides/pg_0010.pdf}
        };
    \end{tikzpicture}
\end{frame}
\note[itemize] {
\item El socket \lstinline[style=normal]!acep! sigue estando disponible para que el servidor acepte a otras conexiones en paralelo mientras antiende a sus clientes (es independiente del socket \lstinline[style=normal]!srv!)
\item Al mismo tiempo, el socket \lstinline[style=normal]!srv! quedo asociado a esa conexi\'on en particular y le permitir\'a al servidor enviar y recibir mensajes de su cliente.
\item Tanto el cliente como el servidor se pueden enviar y recibir mensajes (\lstinline[style=normal]!send!/\lstinline[style=normal]!recv!) entre ellos.
\item Los mensajes/bytes enviados con \lstinline[style=normal]!cli.send! son recibidos por el servidor con \lstinline[style=normal]!srv.recv!.
\item De igual modo el cliente recibe con \lstinline[style=normal]!cli.recv! los bytes enviados por el servidor con \lstinline[style=normal]!srv.send!.
}

\begin{frame}{Envio y recepci\'on de datos}
    \begin{tikzpicture}[remember picture,overlay]
        \node[yshift=-1cm,at=(current page.center)] {
            \includegraphics<1>[width=\paperwidth]{overview_slides/pg_0011.pdf}
        };
    \end{tikzpicture}
\end{frame}
\note[itemize] {
\item El par \lstinline[style=normal]!cli.send!--\lstinline[style=normal]!srv.recv! forma un canal en una direcci\'on mientras que el par \lstinline[style=normal]!srv.send!--\lstinline[style=normal]!cli-recv! forma otro canal en el sentido opuesto.
\item Ambos canales son independientes. Esto se lo conoce como comunicaci\'on Full Duplex
\item TCP garantiza que los bytes enviados llegaran en el mismo orden, sin repeticiones y sin p\'erdidas del otro lado.
\item Otro protocolos como UDP no son tan robustos...
}
\begin{frame}{Envio y recepci\'on de datos en la realidad}
    \begin{tikzpicture}[remember picture,overlay]
        \node[yshift=-1cm,at=(current page.center)] {
            \includegraphics<1>[width=\paperwidth]{overview_slides/pg_0012.pdf}
        };
    \end{tikzpicture}
\end{frame}
\begin{frame}{Envio y recepci\'on de datos en la realidad}
    \begin{tikzpicture}[remember picture,overlay]
        \node[yshift=-1cm,at=(current page.center)] {
            \includegraphics<1>[width=\paperwidth]{overview_slides/pg_0020.pdf}
        };
    \end{tikzpicture}
\end{frame}
\note[itemize] {
\item Sin embargo TCP NO garantiza que todos los bytes pasados a \lstinline[style=normal]!send! se puedan enviar en un solo intento: el programador debera hacer m\'ultiples llamadas a \lstinline[style=normal]!send!.
\item De igual modo, no todo lo enviado sera recibido en una \'unica llamada a \lstinline[style=normal]!recv!: el programador debera hacer m\'ultiples llamadas a \lstinline[style=normal]!recv!.
}

\begin{frame}[fragile]{Envio y recepci\'on de datos}
      \begin{lstlisting}[style=normal]
int s = send(skt,
              buf,
              bytes_to_sent,
              flags           // MSG_NOSIGNAL
            );

int s = recv(skt,
              buf,
              bytes_to_recv,
              flags           // 0
            );

  (s == -1)  // Error inesperado, ver errno
   (s == 0)  // El socket fue cerrado
    (s > 0)  // Ok: s bytes fueron enviados/recibidos
      \end{lstlisting}
\end{frame}
~% endif %~

\begin{frame}[fragile]{recvall: recepci\'on de N bytes exactos}
      \begin{lstlisting}[style=normal]
char buf[MSG_LEN];  // buffer donde guardar los datos
int bytes_recv = 0;

while (MSG_LEN > bytes_recv && skt_still_open) {
  s = recv(skt, &buf[bytes_recv], MSG_LEN - bytes_recv,
                                                0);
  if (s == -1) { // Error inesperado, ver errno
     /* ... */
  }
  else if (s == 0) { // Nos cerraron el socket
     /* ... */
  }
  else {
    bytes_recv += s;
  }
}
      \end{lstlisting}
\end{frame}
\begin{frame}[fragile]{sendall: env\'io de N bytes exactos}
      \begin{lstlisting}[style=normal]
char buf[MSG_LEN];    // buffer con los datos a enviar
int bytes_sent = 0;

while (MSG_LEN > bytes_sent && skt_still_open) {
  s = send(skt, &buf[bytes_sent], MSG_LEN - bytes_sent,
                                              MSG_NOSIGNAL);
  if (s == -1) { // Error inesperado, ver errno
     /* ... */
  }
  else if (s == 0) { // Nos cerraron el socket
     /* ... */
  }
  else {
    bytes_sent += s;
  }
}
      \end{lstlisting}
\end{frame}

~% if interactive is off %~
\subsection{Finalizaci\'on de un canal}
\begin{frame}{Cierre de conexi\'on parcial}
    \begin{tikzpicture}[remember picture,overlay]
        \node[yshift=-1cm,at=(current page.center)] {
            \includegraphics<1>[width=\paperwidth]{overview_slides/pg_0013.pdf}
        };
    \end{tikzpicture}
\end{frame}

\begin{frame}{Cierre de conexi\'on total}
    \begin{tikzpicture}[remember picture,overlay]
        \node[yshift=-1cm,at=(current page.center)] {
            \includegraphics<1>[width=\paperwidth]{overview_slides/pg_0014.pdf}
        };
    \end{tikzpicture}
\end{frame}
\note[itemize] {
\item Parcial en un sentido (envio) \lstinline[style=normal]!SHUT_WR!
\item Parcial en el otro sentido (recepci\'on) \lstinline[style=normal]!SHUT_RD!
\item Total en ambos sentidos \lstinline[style=normal]!SHUT_RDWR!
}

\begin{frame}{Liberaci\'on de los recursos con close}
    \begin{tikzpicture}[remember picture,overlay]
        \node[yshift=-1cm,at=(current page.center)] {
            \includegraphics<1>[width=\paperwidth]{overview_slides/pg_0015.pdf}
        };
    \end{tikzpicture}
\end{frame}

\begin{frame}{Cierre y liberaci\'on del socket aceptador}
    \begin{tikzpicture}[remember picture,overlay]
        \node[yshift=-1cm,at=(current page.center)] {
            \includegraphics<1>[width=\paperwidth]{overview_slides/pg_0016.pdf}
        };
    \end{tikzpicture}
\end{frame}

\begin{frame}{TIME WAIT}
    \begin{tikzpicture}[remember picture,overlay]
        \node[yshift=-1cm,at=(current page.center)] {
            \includegraphics<1>[width=\paperwidth]{overview_slides/pg_0017.pdf}
        };
    \end{tikzpicture}
\end{frame}
~% endif %~

\begin{frame}[fragile]{TIME WAIT -> Reuse Address}
Si el puerto 80 esta en el estado TIME WAIT, esto termina en error (Address Already in Use):
      \begin{lstlisting}[style=normal]
int acep   = socket(...);
int status = bind(acep, ...); //bind al puerto 80
      \end{lstlisting}
\pause
La soluci\'on es configurar al socket aceptador para que pueda reusar la direcci\'on:
      \begin{lstlisting}[style=normal]
int acep   = socket(...);

int val = 1;
setsockopt(acep, SOL_SOCKET, SO_REUSEADDR, &val, sizeof(val));

int status = bind(acep, ...); //bind al puerto 80
      \end{lstlisting}
\end{frame}



~% endmacro %~

~{ content() }~

\end{document}
